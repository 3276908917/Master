\chapter{Introduction, Theory, and Background}

\textcolor{blue}{I have a lot of different important concepts that I need to 
get through, so I can easily imagine this becoming a relatively long 
introduction compared to other master's theses.}

A primary goal of cosmology is to specify, as narrowly as possible, the 
parameters which define our Universe. These include, for example, the overall 
curvature of the Universe as well as its cold dark matter (CDM) content. These
parameters determine the full evolution of the Universe after the inflationary
period (whose beginnings were thought to be non-deterministic)
\cbib{Caravano}. For example, depending on
the makeup of the Universe--how much of its total energy budget exists in the
form of each `ingredient' (cold dark matter, radiation, etc.)--the Universe
can have a finite lifetime. When the proportion of matter is high enough,
gravity will cause the Universe to collapse again on itself. By contrast, if
the proportion of dark energy is high enough, the Universe will continue to
accelerate in its expansion forever.

Cosmology has rapidly evolved into a high-precision science. For example, with
COBE (1989-1993) \cbib{COBE} followed by WMAP (2001-2010) \cbib{WMAP} followed 
by Planck (2009-2013) \cbib{Planck}, the uncertainties on several cosmological 
parameters have been tightened significantly.
\textcolor{red}{Should I try to offer concrete examples of how these missions
increased the precision on our parameter values? I feel like that might not be
a good use of space in the intro.}

Ultimately, the goal of this work is to speed up the kinds of statistical
analyses which are necessary to tighten these uncertainties.
These analyses compare our cosmological observations to what we
would expect to see if we solved the equations of cosmological evolution with
different values for parameters.

\section{Brief Glossary of Our Cosmological Parameters}
\label{sec: param_glossary}

\textcolor{blue}{We need to explain what the different parameters mean! The
big omega terms are likely to be more familiar to readers, we can start with 
those.}

We will get more specific about terms like ``cosmological observations''
and ``what we would expect to see'' in the next section (\ref{sec: Pk_intro}, 
on the matter power spectrum). First, we will briefly introduce concrete
examples of cosmological parameters in which we are interested.

In this work, we will concentrate on different parameters to different 
extents. By the end of this thesis, we will have built and tested two main
emulators, each of which is trained over a different parameter
space.\footnote{Emulators will be described in greater detail in
section~\ref{sec: emulation_intro}. For now, the extremely simple definition 
from the summary will suffice.}
Each emulator accepts as $\matr{X}$ sets of tuples of values for cosmological 
parameters and predicts as $\hat{\matr{Y}}$ the power spectra to which these
tuples correspond. 

%s Introduce the parameter names, THEN talk about what they mean

\textcolor{orange}{This paragraph needs to be rewritten. It's way too messy to 
talk about parameters in order of importance!}

A secondary, ``support'' emulator has been trained over four parameters:
$\omega_b$, $\omega_\text{CDM}$, $n_s$, and $\sigma_{12}$.
The primary emulator for this work has been trained over two additional
parameters: $A_s$ and $\omega_\nu$. Later in this section, we will also 
introduce the terms $h$ and $z$, which are not subject to emulation but which 
will be important to the generation of training data for the emulator (see 
section~\ref{sec: train_emu}. Finally, we will very briefly define terms such 
as $\Omega_k$, $\omega_\text{DE}$, $w_0$, $w_a$ since they make only minor 
appearances in this work.

%s First: elucidate the density parameters

The various $\omega_i$ and $\Omega_i$ terms are all related, and describe the
energy density of the Universe in different energy species.

%s First talk about what \omega and \Omega mean, then talk about what the
%s subscripts mean

%s What do the subscripts mean?

\begin{table}[htb]
\centering
\begin{tabular}{l|l}
\hline
Symbol & Energy Species \\ \hline
$\gamma$ & Relativistic (i.e. radiation, photons) \\
$B$ or $b$ & Baryons \\
$C$, $c$, or CDM & Cold dark matter \\
$\nu$ & Neutrinos \\
$M$ & Matter ($b$, CDM, and $\nu$ together) \\
$K$, $k$, or $\kappa$ & Curvature \\ \hline
\end{tabular}
 \caption[Energy species symbols]{The various energy species into which the 
 	contents of the Universe are categorized, and their conventional symbols.}
 \label{tab: species_symbols}
\end{table}

Table~\ref{tab: species_symbols} gives the various subscripts and the energy species to which
they correspond. We note that $\omega_\gamma$ does not appear in this thesis;
while radiation is critically important in the early stages of the
Universe, it rapidly becomes inconsequential
and radiation-matter equality occurs only 55,000 years after the
beginning of the Universe (\cbib{CO}, page 1194).

While baryons and cold dark matter (CDM) are both pressureless\footnote{This 
is of
course an approximation, but it is a very good one. While a hot baryonic gas 
certainly exerts a pressure, this pressure is negligible in comparison to the energy density of the gas. \textcolor{red}{Where are we getting a unitless
pressure term such that we can compare it in this way to energy density?
Otherwise, terms of different units cannot be compared. I understand that we
can get a unitless energy density by making it relative. Is there a similar
procedure for pressure terms?}}, CDM appears to interact only gravitationally
with other matter. This lack of collisional and radiative self-interaction
prevents CDM clouds from collapsing in the same way that, for example,
baryons can collapse to form stars and planets. Because of the significantly
different behaviors of these two species of matter, we treat them separately
even in cosmological contexts.

{segue from power spectrum discussion?} There is another facet to densities that will be conceptually important to this work: densities $\rho_i$ of different energy species. Similarly to before, it is common to define a related quantity $\Omega_i$ that is normalized, although here the normalization is enforce

\begin{equation}
\sum_i \Omega_i = 1
\end{equation}

IT’s MORE COMPLICATED THAN THAT—TALK ABOUT FLATNESS OF UNIVERSE and critical densities. These $\Omega_i$ parameters are referred to as “fractional energy densities.” The constant of normalization is simply $h^2$. $h$ is known as
the dimensionless Hubble parameter and is merely a different 

\begin{equation}
h = H_0 \, \frac{1}{100 }
\end{equation}

For example, the Planck best-fit value for $H_0$,
$67 \, \frac{\mathrm{km} / \mathrm{s}}{\mathrm{Mpc}}$,
corresponds to the $h$ value $0.67$. 

Unfortunately, this normalization has its disadvantages. WHAT ARE THEY. In particular, in the context of parameter inference, it is inconvenient to try to constrain a term consisting of multiple uncertain parameters. In this case, $h$ is itself a parameter that we would like to constrain, so it would be better to keep it separate. This situation is made more drastic due to the Hubble tension, which leaves the value of $h$ significantly unclear compared to other parameters.

throughout this paper we will refrain from using the conventional
fractional density parameters $\Omega_i$ and use instead the physical density
parameters $\omega_i$.

%s omega_k

According to Andrea P., $\omega_k$ is an evolution parameter only in a narrow 
band around zero--specifically, over roughly the range [-0.05, 0.05].
\textcolor{red}{By how much does evolution mapping fail here? And, if 
$\omega_k$ is truly a shape parameter, why does Ariel's FECS call $\omega_k$
an evolution parameter?} We do not emulate over $\omega_k$ but assume take
it to be zero for all cells. However, as we explain in
section~\ref{sec: ev_mapping_intro}, this does not limit the applicability of
our emulators. The use of nonzero $\omega_k$ values will simply entail a
relabeling of the emulated power spectra. \textcolor{orange}{Then again, I'm
holding $\sigma_{12}$ fixed, so does it change anything at all?}

%s n_s

The parameter $n_s$ is called the spectral index and describes the power laws
to which the power spectrum may be fit. We will delay its explanation to
section~\ref{sec: Pk_intro}, where we introduce the power spectrum.

%s sigma12 versus sigma8. This should come last as it is by far the most
%s challenging of the parameters to explain.

\textcolor{orange}{We should also have the $\sigma_8$ section here! This 
section will include an extremely brief summary of Ariel's paper motivating 
the use of $\sigma_{12}$ instead of $\sigma_8$.}

When comparing $\sigma_8$ results from different analyses, for example, the significance of the parameter itself is easily lost. Whenever two analyses differ, even slightly, in their values for $h$, the meaning of sigma 8 changes!

Since $h$ is already its own parameter, the conventional $\sigma_8$ parameter 
is truly a mixture of two parameters. This presents a host of misleading 
results and statistical ambiguities (\cbib{San20}) which are outside of the 
scope of this work (\textcolor{orange}{Nevertheless I would like to summarize 
a couple of key arguments}) but which prompt us to abandon $\sigma_8$.

In the end, however, the argument in favor of $\sigma_{12}$ most relevant to
this paper is the evolution mapping argument. We can greatly simplify the
parameter space of emulators with the help of $\sigma_{12}$, and $\sigma_8$
is simply incapable of exploiting the parameter degeneracies to the same
extent.
 

\section{The Matter Power Spectrum}
\label{sec: Pk_intro}

%%%%%%%%%%%%%%%%%%%%%%%%%%%%%%
%%  Einbinden einer Grafik  %%
%%%%%%%%%%%%%%%%%%%%%%%%%%%%%%

\begin{figure}[htb]
  \centering
  \includegraphics[scale=0.5]{siegel}
  \caption[Kurzform f"ur das Abbildungsverzeichnis]{Dies ist die Erkl"arung zum Bild.}
\end{figure}

How can we describe the Universe in such 
a way as to allow quantitative definitions of these phrases? As a starting 
point, we can imagine quantifying the energy density $\rho(\bm{x})$ of the
Universe at any one point $\bm{x}$ in three-dimensional space.

Let us return to the idea that cosmological parameters determine the the 
evolution of the Universe.
% In what ways can we quantify the Universe, in order to be able to describe
% its evolution? We can simply consider the time dependence of the density.

We can define a similar quantity, a relative density:

\begin{equation}
\delta(\bm{x}) = \frac{\rho(\bm{x}) - \bar{\rho}}{\bar{\rho}}
\end{equation}

which we refer to as the ``matter density contrast field.'' $\bar{\rho}$
represents the average energy density of the entire Universe.

$\delta(\bm{x})$ easier than $\rho(\bm{x})$ to work with since it
is unitless and especially because it is normalized--the integral of the
matter density contrast field, taken over the entire Universe, is unity. One particularly popular metric is the cosmic matter density 
contrast field. evolution of the Universe? The various constituents of the Universe 

Next, I want to talk about one way of describing the matter density contrast 
field: the power spectrum. The power spectrum can be probed in many different 
ways, and its precise shape and amplitude can tell us about several of these 
cosmological parameters.

Actually, the power spectra we are discussing in this thesis are linear-theory
power spectra of non-neutrino matter. \textcolor{blue}{But anyway, here I will cover some of 
the tried-and-true basic explanations of what the power spectrum is and why it 
is interesting for the question of parameter inference. I also want to 
discuss: why do we care about the linear-theory power spectrum? Why not jump 
straight to the nonlinear case?}

\textcolor{red}{The argument on the exponential should be a dot product, 
right?}

\begin{equation}
\tilde{\delta} (\bm{k}) = \int d^3 x \delta(\bm{x}) \exp(-i \bm{k} \bm{x})
\end{equation}

\begin{equation}
\langle \tilde{\delta} (\bm{k}_1) \tilde{\delta} (\bm{k}_2)^* \rangle
=
(2 \pi)^3 \delta_D^{(3)} (\bm{k}_2 \ \bm{k}_1) P(\bm{k})
\end{equation}

%s Why do we care?

The power spectrum is of great importance to parameter inference because of
its unique dependence on certain parameters.
% Here: refer to specific P(k) equations where omega_b and n_s showed up,
% turnover scale.
In section~\ref{sec: boltzmann_intro}, we will include several figures 
illustrating more clearly what we mean by this.

%s What is the BAO, what is the turnover scale

\textcolor{orange}{WE SHOULD DEFINITELY incorporate some of ARIEL'S DISCUSSION 
OF THE BAO 
(``WHY ARE THERE
WIGGLES IN THE POWER SPECTRUM'') AND WE SHOULD DO ENOUGH OF THE MODE / HORIZON
DISCUSSION TO ELUCIDATE THIS CONCEPT OF A TURNOVER SCALE!!!}
\cbib{FECS}

%s Everyone else uses h units. Why don't we?

\textcolor{blue}{This section will also reiterate some parts of 
section~\ref{sec: param_glossary} and further develop our moving-away from 
$h$, this time in terms of units on power spectra plots. explain the unit system
    we are using (ditch $h$ factor because it messes up everything--but
    only briefly summarize the main arguments of Sanchez 2020)}

%! Try to more carefully explain why it doesn't work: h is not a parameter
% over which we emulate--sigma12 is. Therefore we cannot show h in the final
% plots. This is a much subtler issue than the h complaints in the
% previous section.

Conventional emulator calibration entails the historical units of Mpc / $h$,
but if we use instead units of Mpc, then we can distill all of the evolution
parameters into one parameter, $\sigma_{12}$. 


\section{Boltzmann Solvers and CAMB}
\label{sec: boltzmann_intro}

I want to talk about what a Boltzmann solver is and what kinds of equations it is solving.

To hint at what's to come, I start off this section by noting that several cosmological parameters have a fairly unique impact on the shape of the power spectrum, while others have a degenerate impact. Wouldn't it be great if we could know what the power spectrum would look like if we increased parameter $x$? Boltzmann solvers can help us with that.

% Examples of great results from Boltzmann solvers

Three shape parameters of core interest to this paper are the physical density in baryons $\omega_b$ (whose impact on the power spectrum is shown in figure~\ref{fig: omega_b_dependence}), the physical density in cold dark matter $\omega_c$ (figure~\ref{fig: omega_c_dependence}), and the spectral index $n_s$, (figure~\ref{fig: ns_dependence}). The remaining parameters $\sigma_{12}$ and $A_s$, as well as the quantities $z$ and $h$, all shift only the amplitude of the power spectrum. We show the amplitude shift associated with various $\sigma_{12}$ values in figure~\ref{fig: sig12_dependence} and stress that the same plot only needs to be relabeled in order to illustrate the impact of $A_s$, $z$, $h$, etc. 

% What are they solving?

This will mostly just be a theoretical discussion of these solvers. The hands-on stuff comes in the non-introductory section on CAMB.

%%%

In essence, Boltzmann codes solve XXX in order to give us the power spectrum
of any universe characterized by some set of cosmological parameters. For
example, figure~\ref{fig: vary_omega_b} shows the impact of varying the
physical density in baryons, $\omega_b$. 

Several parameters have fairly unique impacts on the power spectrum.
Therefore, we can imagine building a collection of power spectra labeled by
their parameter configurations and comparing our real-world observations to
them. This should allow us to perform parameter inference.

To conclude this section, we make mention of three specific Boltzmann solvers: 
CLASS, CAMB, and CMBFast.
CAMB was written in Fortran and comes with a Python wrapper which we
will use as a starting point (along with GPy) for our Python package
Cassandra-Linear.

CLASS was written in C and also comes with a Python wrapper.
%I will briefly justify our use of CAMB over CLASS.
% Most of our existing emulator pipeline (including the emulator 
% COMET\footnote{\textcolor{green} {Double check this, does COMET actually 
% rely on CAMB?}}) DOES THIS EVEN MATTER?
For this project, we elected to continue with CAMB as the scientists at
OPINAS at the MPE were better equipped to provide support with CAMB.
\textcolor{green}{Furthermore, the CLASS documentation
is not nearly as strong as it is with CAMB, and we already encountered
extreme difficulty simply in recreating results already previously obtained
via CAMB!}

\section{Monte Carlo Markov Chains}

This can be a very brief section, but I want to discuss a little bit of how most modern parameter inference works because it motivates the need for extremely fast power spectrum computation. It provides a sort of conceptual bridge between our ``pure'' goal (quantifying the cosmos) and the nitty-gritty bulk of the paper (optimizing emulator performance).

Metropolis-Hastings algorithm.

We don't know what the true probability distribution of power spectra is. In order to build this distribution with simulation results, we simply draw from the distribution. \textcolor{orange}{Refer to ``Data to Insights'' lecture notes in order to tighten this description.}

\section{Emulation: Basic Principles}
\label{sec: emulation_intro}

To conduct these MCMC analyses, we need several thousands of power spectra. However, if our Boltzmann solvers take on the order of three seconds to run, then these solvers will become the bottleneck of our analysis. \textcolor{orange}{Give some specific numbers for this.}

This motivates the introduction of emulation, basically multi-dimensional interpolation, in order to predict the power spectra. These predictions are orders of magnitude less time-expensive. 

Emulators interpolate across a high-dimensional parameter space. The primary
limitation is that the emulator has to be built with every possible parameter
in mind that an end-user could wish to vary. Yet there is a large number of
different cosmological parameters discussed in the modern literature.
``Currently available emulators only sample a few cosmological parameters,
often with restrictive ranges, and are not applicable to more general
parameter
spaces'' (\cbib{San21}). ``Due to the high computational cost of the required
simulations, [...] current emulators leave out parameters such as the
curvature
of the Universe or dynamic energy models beyond the standard CPL
parametrization'' (\cbib{San21}).

I'll talk a little about different emulators currently available, such as COMET. Some emulate non-linear power spectra, for example, and several even include massive neutrinos. But this thesis will demonstrate that massive neutrinos can be included into our evolution mapping approach, which will be introduced in section~\ref{sec: ev_mapping_intro}.

% (This is good news because the evolution mapping approach greatly simplifies the parameter space, and enhances the accuracy), which is the subject of the next section.

\section{Gaussian Process Regression}
\label{sec: gpr_intro}

% What is a Gaussian Process?

Most emulators are based on a Gaussian Process (GP). A GP is a Gaussian
distribution over functions\footnote
{A GP is the limit of a one-hidden-layer neural network as the number of
neurons approaches infinity.}, which can be interpreted
as the infinite-dimensional generalization of the multivariate normal
distribution. The inference of continuous values with a GP prior
is known as Gaussian process regression, or Kriging. GP regression is a
powerful non-linear multivariate interpolation tool. The computational
complexity of inference and likelihood evaluation within GP regression is cubic
in the number of points. This makes GP regression an excellent companion to
Latin hypercube sampling (LHS), which makes highly efficient use of a limited 
number of samples and whose basic idea will be explained in section~\ref{sec:
lhc_theory}.

Neural networks (NNs) generally need much larger sample sizes to reach
comparable levels of
accuracy. Due to various alterations in the Cassandra-Linear code over its
development, several regenerations of the various emulator data sets were
necessary. This practical constraint motivated the use of a GP for our
emulator. Furthermore, NNs invariably require much more complicated setup and
tuning--for example, in the precise architecture of the network (e.g. nodes
per layer, layer types) as well as the hyperparameters (e.g. learning rate).
By contrast, as we explain in section~\ref{sec: train_emu}, a Gaussian
process regression is highly straightforward to set up and modify. Therefore,
for a demonstration project such as Cassandra-Linear, we elected to base our
emulator on a GP. Please refer to the section~\ref{sec: future_work} for a
continuation of this discussion.

Are there other prediction approaches besides GPs and NNs? IF so, I need to
further justify WHY we’re using GPs.
GPs work best when there are few samples and a lot of parameters, right?
But why is that so? What is the math behind that?


\section{Sampling Approach: Latin Hypercube}
\label{sec: lhc_theory}

I imagine this is going to be an extremely short section. We should motivate why we're using this style of sampling.

What is the theoretical best LHC that we could make?

Besides, can we explain this equation?


\section{Evolution Mapping}
\label{sec: ev_mapping_intro}

\textcolor{blue}{I want to briefly summarize why we can funnel all of the 
evolution parameters through $\sigma_{12}$ in this way. This objective may be
too ambitious for this paper, because I would have to go through each
evolution parameter and derive its evolution nature in a few lines of 
equations.} \textcolor{orange}{Maybe I can just do the same thing as Ariel
recommended with the GPR kernel: ``we played around and found that this
works.''}

(\cbib{San21}) proposes to divide up the full set of cosmological
parameters into two categories: \textit{evolution} parameters $\mathcal{O}_E$
(such as $\omega_b$, $\omega_c$, and $\eta_s$)
affect the amplitude of the power spectrum at a particular redshift, while
\textit{shape} parameters $\mathcal{O}_S$
(such as $\omega_K$, $\omega_\text{DE}$, w(a))
affect the shape of the power
spectrum.

We take, as the evolution mapping relation for the power spectrum, equation 13
from \cbib{San21}:

\begin{equation}
\label{eq: evMapping_pSpectrum}
    \Delta^2_L (k | z, \Theta_s, \Theta_e)
    =
    \Delta_L^2 (k | \Theta_s, \sigma_{12} \left( z, \Theta_s, \Theta_e \right))
\end{equation}\footnote{Varying $z$ has the same effect as varying an
evolution parameter, which is why it appears on the RHS only as an argument to
the $\sigma_{12}$ ``function.'' We write it separately from $\Theta_e$ to
emphasize that $z$ does not describe a property of the Universe, but is
simply used as a proxy here for \textcolor{red}{conformal?} time
\textcolor{red}{elapsed since the Big Bang? (but we can only observe up to
$z = 1100$...)}.}

Why is this scheme important? Evolution mapping greatly simplifies the emulator
implementation. Because we can
funnel all of the evolution parameters through $\sigma_{12}$, we've effectively
collapsed an entire category of parameters to just one parameter. Fewer
parameters means that we get a more accurate emulator.

``At the linear level, all models characterized by identical shape parameters
and the same values of the parameter combinations $b \sigma_{12}(z)$ and
$f \sigma_{12}(z)$ will be identical'' (\cbib{San21}).

Now, for the hiccup, which segues into the next section: this scheme is broken by one parameter, the Universe's
density in neutrinos. (In the next section: why this is so and what we can do
about it.)


\section{Neutrinos and Their Cosmological Impact}

(\cbib{Kiakotou}): ``Neutrinos with masses on the eV scale or below will be a
hot component of the dark matter and will free-stream out of overdensities and
thus wipe out small-scale structures.''

``In general, a larger density of relativistic species leads to a smaller
growth of matter fluctuations'' (\cbib{Zennaro}).

The point of this section is: why is $\omega_\nu$ bad for the
evolution mapping scheme? Because neutrinos exhibit redshift-dependent
damping of the power-spectrum, and therefore affect both the shape and the
amplitude of the power spectrum. Whenever massive neutrinos are present,
the growth factor becomes scale-dependent, which disrupts the
evolution-mapping scheme.

Why do they behave in this way? All neutrinos start off as
relativistic particles in the early Universe, acting as a type of radiation.
But as the Universe continues to expand and cool, the neutrinos behave
increasingly like dark matter.
In this way, the physical density in neutrinos impacts both the shape and the
evolution.

``The popular heuristic formula for the linear theory suppression of the matter
fluctuations by free-streaming $\nu$, $\Delta P(k) / P(k) \approx -8 f_\nu$, is
valid only on very small scales $k > 0.8 h$ / Mpc, However, it is not of
practical use as this is in the strongly nonlinear regime of matter
clustering'' (\cbib{Kiakotou}).

One proposed solution is to treat the neutrinos as a small correction factor
to the results from an anologous cosmology with the same $\omega_m$ but with
$\omega_\nu = 0$. This of course limits the applicability of our emulator to
cosmologies with very small $\omega_\nu$, but this constraint agrees with
current observations (\textcolor{orange}{which?}).

I want to end this section with a vague plan of action: we want to play around with CAMB power spectra to see if there are any simple ways around this limitation in our approach.

%%% New stuff

% what is a MEMNeC?

We already have an approximation for the power spectrum of a massive-neutrino cosmology within the evolution mapping scheme. The $\sigma_{12}$ value that we described earlier is actually the $\sigma_{12}$ value of the model's MEMNeC. can already be approximated within evolution-mapping by slightly altering scheme. \textcolor{red}{Is it fair to say we are adjusting, or was this actually the same scheme as it always was?} We take a MEMNeC and the desired cosmology. The sigma 12 is actually the sigma 12 of the MEMNeC. Then we treat the physical density in neutrinos as a shape parameter along with $A_s$.
