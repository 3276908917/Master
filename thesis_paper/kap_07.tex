\chapter{Discussion and Conclusion}
\label{chap: disc_and_conc}

\section{Tightness of the Priors Used}
\label{sec: priors}

With this section I would like to revisit the specific values for the priors,
that I only briefly mentioned back in section~\ref{sec: build_lhc}.

First of all, from a purely practical consideration, expanding the priors was
not feasible due to the high incidence of unsolvable cells.

But second, this may not be a significant limitation to the utility of the
emulators introduced here, because they are already quite wide compared to
current state-of-the-art parameter inferences. \textcolor{green}{CITATIONS}.

As of 19.06.23, ``COMET'' is the default for the emulator. It is the most 
restrictive of the three options and was implemented in order to totally 
eliminate the problem of unsolvable cells, allowing us to train our 
demonstration emulator over an LHS without any significant gaps.

Our hope was that the narrow parameter ranges would furthermore help the 
demonstration emulator to achieve high accuracy--in principal, success here
means that we can simply ``scale up'' the approach of this work by
simultaneously expanding the priors as well as the total number of training
samples. Unfortunately, it is not clear if we can scale up the emulator past
the point at which unsolvable cells begin to appear. Since Latin hypercube
sampling is designed to evenly sample a space, unsolvable cells certainly
indicate that parts of the parameter space lack representation in the training
data. In these regions, our emulator will be forced to interpolate across
large gaps, or worse, extrapolate (if the unsolvable cells occur at the edges
of the parameter space \textcolor{orange}{This is something that I should have
shown... i.e. with plots}).
% Andrea recommends a plot coloring points by the "extremeness" in our sample
% space.

We predict that increasing the total number of cells will only marginally
reduce the issue of unsolvable cells \textcolor{orange}{This is something that 
I should have shown... i.e. with plots}). We can imagine the subspace of
solvable points as some hypervolume within a hypercube determined by our
priors, and the emulator's training coverage as an approximation of this 
hypervolume with small hypercubes whose size is determined by the separation 
between points in the sample. In the ideal case, the space of solvable points 
is the same as the Latin hypercube. When this is not so, we can at least
reduce the error associated with our approximation of the space of solvable
points by shrinking the hypercubes we use in our approximation (i.e. by
increasing the total number of points in our sample). \textcolor{orange}{
To give a sense of the marginal nature of this error reduction, we can
consider how small our hypercubes already are. For simplicity, let's examine
just one axis of the hypercube. With 5000 samples in the ``MEGA'' priors,
the length of the training coverage MOST STRONGLY DETERMINED BY ONE HYPERCUBE
IS: UNFINISHED THOUGHT}

It seems reasonable to think that unsolvable cells indicate extreme regions of 
the parameter space, rather than isolated holes. Therefore, it would be 
misleading to claim that the final ``MEGA'' emulator corresponds to, for 
example, any prior ranges in table 00A; in truth, the emulator would 
correspond to a potentially (this is a dangerous word and opens you up to hard 
questions) complicated shape inscribed within the six-dimensional rectangular 
hyperprism.

\section{Minimum Separation of the Training LHC}
\label{sec: error_from_lhc}

What is the impact of the minimum separation? Surely the minimum separation
should be a proxy for the evenness of the coverage of the space of
cosmologies. Therefore, we expect the error variance to increase much more
dramatically than, say, the average bias.

How would we
be able to quantify the error due to this? We could try to compare the
emulator performance trained on hyper cubes of various minimum distances.

\section{Resolution of the k Axis}

This might go better in the CassL section, but I think I ought to motivate the decision to use length-300 arrays.

\section{Number of Training Samples}
\label{sec: num_samples}

\textcolor{blue}{Justify choice of 5000 samples for each: maybe we can make a
trend plot showing diminishing returns in test error?}

% This might go better in the CassL section, but I think I ought to motivate the decision to use 5000 training arrays.

\textcolor{orange}{I'll have to concede that the results of this section are not entirely comprehensive; we didn't train any emulators over the uncertainties of analogous validation hypercubes. All comparisons here use the simpler pipeline of just two data sets, training and testing.}


\section{Linear Sampling in Different Parameters}

We also tried sampling in $\sigma_{12}^2$ as well as $\sqrt{\sigma_{12}}$.
Unfortunately, we were unable to conclude anything about the effectiveness of
these strategies--there appears to have been some mistake in our code, such
that the errors are much larger than can be explained on account of poor
sampling.

See figures~\ref{fig: sigsquare_sample} and~\ref{fig: sigroot_sample} for
illustrations of the problem. In a future work, it would be helpful to
investigate these problems further. We may find that a different sampling
strategy will more efficiently reduce the deltas that we see in our emulator.

\section{Summary of the Paper}

What was the main objective of this thesis? What were the key results of this work? Why are they important? How do these results compare with results from papers on similar subjects?

\section{Future Work}
\label{sec: future_work}

Here I will talk about what kinds of questions we estimate will be most fruitful for further inquiries about this topic and this code.

Is there a theoretically perfect LHC generator?

In order to expand the priors, it would be helpful to investigate why CAMB does not allow negative redshifts, in case this can be adapted. Alternatively, CLASS \textcolor{green}{allows} negative redshifts, so it may be worthwhile to repeat the work of this thesis using power spectra from CLASS. This would involve familiarization with a new platform, however, and so this escapes the scope of this thesis. Remember that negative redshifts would be a helpful feature for this work because it would allow us to investigate much broader priors.

Should we use a neural network instead of a Gaussian process for our emulator? AndreaP and Alex Eggemeier are already on the job: they are converting COMET to a neural network approach. We recommend that the reader follow future COMET papers for investigations into this question.

GP's allow natural propagation of
uncertainty in predictions to the final posterior distribution; neural
networks lack this feature. At the same time, NNs provide larger speedups \textcolor{green}{CITATIONS}.
