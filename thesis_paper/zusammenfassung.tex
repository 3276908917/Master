\addcontentsline{toc}{chapter}{\protect Abstract}


\chapter*{Abstract}

\begin{comment}
The color code in this document: \textcolor{blue}{Explanations to the
proofreader; we'll simply cut these out shortly before submission.}
\textcolor{orange}{Notes to myself--I don't need input here, I just need time
to plan and execute.} \textcolor{green}{I need to add citations / 
justification for the claims here, either by generating more plots or by going
back and looking through the papers I used to get started with this project.}
\textcolor{red}{I have some confusion about this area and I would appreciate
feedback from the proofreader.}
\end{comment}

\textcolor{red}{I feel like the extreme simplifications in the next paragraph
would be helpful for someone with no idea about this field, but what do you
think? Maybe they would be more appropriate for easing the reader into
particular sections on these topics?}

The power spectrum can be thought of as a statistical description of the way 
matter clumps together in the Universe.
Comparison of galaxy redshift survey observations with theoretical predictions 
for the cold-matter linear-theory power spectrum $P_L (k)$ establish
increasingly tight estimates of various parameters describing our Universe.

In order to facilitate these comparisons, we an emulator of $P_L (k)$
that greatly decreases the computational cost. An emulator is a multi-
dimensional function produced by interpolation across many training points $
(x, y)$.

Evolution mapping is a technique for simplifying the parameter space via
exploitation of a degeneracy in the impact of several cosmological parameters
on the power spectrum. The evolution mapping scheme of \cbib{San21} has 
already achieved success in emulators produced by the COMET team.
However, these emulators exclude massive neutrinos
(setting the physical density of the Universe in massive neutrinos to zero)
as they present a challenge to the evolution mapping framework.

This work seeks to extend the evolution mapping scheme to massive-neutrino 
cosmologies through two primary modifications: replacement of $\sigma_{12}$
with $\tilde{\sigma}_{12}$ in the evolution mapping relation, and
recategorization of the scalar mode amplitude $A_s$ as a shape parameter.
We find that $A_s$, which is traditionally treated as an evolution parameter, 
can be used to quantify the suppression of structure 
growth due to massive neutrinos.

We introduce a new Python package, Cassandra-Linear, which implements these
modifications to produce emulators of $P_L(k)$. We include various error 
statistics and show that the for the vast majority of tested cosmologies, the 
emulator performs within the 0.1\% \cbib{Seljak} error bounds quoted about 
CAMB.

\textcolor{orange}{Why does this work? Before, the shape impact of
$\omega_\nu$ 
was $z$-dependent. In other words, if we held $\omega_nu$ constant and varied
$z$, the power spectrum would not vary only in overall amplitude. However, if
we fix $\omega_\nu$ and $A_s$, $z$ becomes an evolution parameter again. Why
does fixing $A_s$ accomplish this? That's something for the theorists to 
figure out!}
