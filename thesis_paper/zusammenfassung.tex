\addcontentsline{toc}{chapter}{\protect Zusammenfassung}


\chapter*{Zusammenfassung}

The color code in this document: \textcolor{blue}{Explanations to the
proofreader; we'll simply cut these out shortly before submission.}
\textcolor{orange}{Notes to myself--I already know what I would like to do to
enhance these areas.} \textcolor{green}{I need to add citations / 
justification for the claims here, either by generating more plots or by going
back and looking through the papers I used to get started with this project.}
\textcolor{red}{I have some confusion about this area and I would appreciate
feedback from the proofreader.}

\textcolor{red}{I feel like the extreme simplifications in the next paragraph
would be helpful for someone with no idea about this field, but what do you
think? Maybe they would be more appropriate for easing the reader into
particular sections on these topics?}

``The goal of this project is to produce an emulator of the linear-theory
power spectrum based on evolution mapping'' (A. G. S\'{a}nchez, private
communication). An emulator is a multi-dimensional function
produced by interpolation across many training points $(x, y)$. The power
spectrum can be thought of as a statistical description of the ``clumpiness''
of matter in the Universe. Evolution mapping is a technique for simplifying
the parameter space by identifying many cosmological parameters as basically
the same in their impact on the power spectrum.

The evolution mapping scheme of \cbib{San21} has already achieved success in
emulators of the nonlinear power spectrum. However, massive neutrinos do not
fit neatly into the scheme. Consequently, emulators implementing evolution
mapping have typically fixed the physical density of the Universe in 
neutrinos, $\omega_\nu$, to zero.

This work seeks to extend the evolution mapping scheme to massive-neutrino 
cosmologies through expansion of the parameter space over which the emulator
is trained. In particular, we find that the scalar mode amplitude $A_s$, which
is traditionally treated as an evolution parameter and therefore excluded from 
the parameter space. can be used to quantify the suppression of structure 
growth due to massive neutrinos.

We find that, by including $A_s$ in our parameter space, we can successfully 
train over the physical density of the universe in massive neutrinos. We introduce a new emulation code,
Cassandra-Linear, which combines this expanded parameter space with evolution
mapping. We include various error statistics and show that the emulator
performs roughly at the level of error associated with CAMB itself
\textcolor{green}{Do I have a citation for the error associated with CAMB?}

\textcolor{orange}{Why does this work? Before, the shape impact of
$\omega_\nu$ 
was $z$-dependent. In other words, if we held $\omega_nu$ constant and varied
$z$, the power spectrum would not vary only in overall amplitude. However, if
we fix $\omega_\nu$ and $A_s$, $z$ becomes an evolution parameter again. Why
does fixing $A_s$ accomplish this? That's something for the theorists to 
figure out!}
