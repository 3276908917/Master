\addcontentsline{toc}{chapter}{\protect Abstract}


\chapter*{Abstract}

\begin{comment}
The color code in this document: \textcolor{blue}{Explanations to the
proofreader; we'll simply cut these out shortly before submission.}
\textcolor{orange}{Notes to myself--I don't need input here, I just need time
to plan and execute.} \textcolor{green}{I need to add citations / 
justification for the claims here, either by generating more plots or by going
back and looking through the papers I used to get started with this project.}
\textcolor{red}{I have some confusion about this area and I would appreciate
feedback from the proofreader.}
\end{comment}

The cold-matter linear-theory power spectrum $P_L (k)$ can be thought of as
a statistical description of the way matter clumps together in the Universe.
Comparison of new galaxy redshift survey observations with theoretical 
predictions of $P_L(k)$ help to establish increasingly tight estimates of 
various cosmological parameters.

In order to facilitate these comparisons, we apply emulation to greatly
decrease the computational cost of evaluating $P_L (k)$ for any given
cosmology. An emulator is a multi-dimensional function produced by 
interpolation across many training points $(x, y)$.

Evolution mapping is a technique for simplifying the parameter space via
exploitation of a degeneracy in the impact of several cosmological parameters
on the power spectrum. The COMET team has already achieved successed in
applying the evolution mapping scheme of \citet{San21} to the problem of
emulation. However, these emulators assume that the physical density of the
Universe in massive neutrinos is $\omega_\nu = 0$. Massive neutrinos are 
excluded because they challenge the evolution mapping framework.

This work seeks to extend the evolution mapping scheme to massive-neutrino 
cosmologies through two primary modifications. First, for any given cosmology,
we replace $\sigma_{12}$ in the evolution mapping relation with
$\tilde{\sigma}_{12}$; this is the $\sigma_{12}$ value of the cosmology's
matter-equivalent massive-neutrino cosmology (MEMNeC).
Second, we recategorize the scalar mode amplitude $A_s$ as a shape parameter.
We find that $A_s$, which is traditionally treated as an evolution parameter, 
can be used to quantify the suppression of structure growth due to massive 
neutrinos.

We introduce a new Python package, Cassandra-Linear, which implements these
modifications to produce emulators of $P_L(k)$. We include various error 
statistics and show that the for the vast majority of tested cosmologies, the 
emulator performs within the 0.1\% error bounds quoted about 
CAMB \cbib{Seljak}.

\selfcomment{Why does this work? Before, the shape impact of
$\omega_\nu$ 
was $z$-dependent. In other words, if we held $\omega_nu$ constant and varied
$z$, the power spectrum would not vary only in overall amplitude. However, if
we fix $\omega_\nu$ and $A_s$, $z$ becomes an evolution parameter again. Why
does fixing $A_s$ accomplish this? That's something for the theorists to 
figure out!}
