\addcontentsline{toc}{chapter}{\protect Zusammenfassung}


\chapter*{Zusammenfassung}

The color code in this document: \textcolor{blue}{Explanations to the
proofreader; we'll simply cut these out shortly before submission.}
\textcolor{orange}{Notes to myself--I already know what I would like to do to
enhance these areas.} \textcolor{green}{I need to add citations / 
justification for the claims here, either by generating more plots or by going
back and looking through the papers I used to get started with this project.}
\textcolor{red}{I have some confusion about this area and I would appreciate
feedback from the proofreader.}

\textcolor{red}{I feel like the extreme simplifications in the next paragraph
would be helpful for someone with no idea about this field, but what do you
think? Maybe they would be more appropriate for easing the reader into
particular sections on these topics?}

``The goal of this project is to produce an emulator of the linear-theory
power spectrum based on evolution mapping'' (A. G. S\'{a}nchez, private
communication). An emulator can be thought of as a multi-dimensional function
produced by interpolation across many training points $(x, y)$. The power
spectrum can be thought of as a statistical description of the ``clumpiness''
of matter in the Universe. Evolution mapping is a technique for simplifying
the parameter space by identifying many cosmological parameters as basically
the same in their impact on the power spectrum.

The evolution mapping scheme of \cbib{San21} has already achieved success in
emulators of the nonlinear power spectrum. However, evolution mapping has 

his work seeks to extend the
evolution mapping scheme to massive-neutrino cosmologies by applying a
correction
factor to results from emulators built on massless-neutrino simulations. We
find that the scalar mode amplitude $A_s$ can be used to quantify the
suppression of structure growth due to massive neutrinos. Consequently, by
including this parameter, we can successfully train over the physical density
of the universe in massive neutrinos. We introduce a new emulation code,
Cassandra-Linear, which combines this expanded parameter space with evolution
mapping. We include various error statistics and show that the emulator
performs roughly at the level of error associated with CAMB itself
\textcolor{green}{Do I have a citation for the error associated with CAMB?}