\chapter{Conceptual Outline of Our Emulator Code and Pipeline}

\textcolor{blue}{The point of this section is to introduce the reader to the
basic structure of the emulator pipeline. Most of the discussion will follow
from a flow chart showing the various conceptual pieces of the emulator. Also
in this section, I will introduce various choices made (e.g. 5000 training
spectra, 300 points for each spectrum) WITHOUT justifying any of them--later,
in chapter~\ref{chap: disc_and_conc}, we'll justify these decisions and
talk about the consequences of alternate settings.}

In order to simplify the process of testing different emulator setups, and to make power
spectra more accessible to beginners in the field, we have developed a Python package,
which we call Cassandra-Linear (CL)1. CL represents a full emulation pipeline from the
generation of training and testing data to the plotting of error statistics associated with
the emulation.

% We're gonna need a flow chart for this puppy. That should be priority number
% one, because the writing should constantly refer back to it.

\section{Design Choices}

By default, each Latin hypercube consists of $N_s = 5000$ entries. For 
justification of this quantity, please refer to
section~\ref{sec: num_samples}.

$N_k = 300$.

\section{Creation of Separate Emulators}
\label{sec: 2emu_intro}

\textcolor{blue}{This is a brief introductory section motivating the creation 
of two emulators:
unlike the other five parameters, the physical density in massive neutrinos 
has a hard lower bound. This creates some minor regression problems (appeal to 
boundary effects / danger of extrapolation).}

One potential weakness of \textcolor{orange}{this} emulator outline emerges 
from the prior range for the physical density in neutrinos. The lower bound 
($\omega_\nu = 0$) of this range for the physical density in neutrinos is 
highly firm; any negative value would be totally unphysical. However, this 
lower bound is also inclusive in the sense that the massless neutrino case is 
still very much of interest to us. GP predictions are, naturally, at their 
strongest in the regions densest with training samples. Since the massless 
neutrino case represents the end of a parameter range, the sampling must be 
less dense there. Indeed, no massless-neutrino cases are even in our training 
sample, XXX being the minimum value of $\omega_\nu$ in our 
LHS.\footnote{\textcolor{orange}{What should I say to people who complain: 
``why not just manually add massless-neutrino samples to the original training 
set? Why not simply have one emulator trained over a more diverse training 
set?'' I think it would be better if we simply focus on the $\omega_\nu = 0$
case not being adequately captured by interpolation, since we don't have any
samples on the ``left'' side of $\omega_\nu = 0$.}}
Therefore, the massless-neutrino case is actually \textcolor{green}{a slight
extrapolation} from our training data, which is dangerous for accuracy.

To 
address this unique problem among our parameter ranges, CL can automatically 
access one of two emulators depending on the user’s input. The primary 
emulator is trained over the aforementioned prior range in $\omega_\nu$
(table~\ref{tab: neutrino_priors}). In addition to this, we’ve trained a 
separate emulator specifically to handle the massless neutrino case. In 
practice, this entails the same prior ranges for the parameters $\omega_b$, 
$\omega_c$, $n_s$, and $\sigma_{12}$. However, here we no longer need $A_s$;
as discussed in chapter~\ref{chap: A_s}, $A_s$ helps to characterize the
structure growth suppression induced by massive neutrinos. Besides this, for 
the purposes of our training and validation data, $A_s$ is redundant since it
is otherwise an evolution parameter, and we are already specifying the 
amplitude of the power spectrum through sigma12. 

To train this massless-neutrino emulator, we use the same code as described in 
in this chapter and chapter~\ref{chap: implementation}. Since the
massless-neutrino emulator is trained over just four parameters, and since we 
nevertheless continue to train over 5000 samples, we expect the
massless-neutrino emulator to be universally more accurate. However, as
discussed and plotted in section~\ref{sec: num_samples}, our decision to train 
over 5000 is justified by the rapidly diminishing returns in additional 
accuracy.

\textcolor{orange}{The following paragraph is bad. I'm not sure yet if I can
salvage anything. The claims are not true and the work is not yet done.}

Therefore, the reason why the massless-neutrino emulator is more 
accurate is because massive neutrinos in particular have a highly complicated 
impact on the power spectrum. Even with the inclusion of the $\omega_\nu$ and 
$A_s$ parameters, our investigations in chapter~\ref{chap: A_s} reveal that
these two parameters may not be sufficient to fully characterize the 
suppression of the power spectrum by massive neutrinos. Of course, we would 
like to reiterate that, though we have applied a symbolic regression approach 
in order to independently justify our analytical expression from
chapter~\ref{chap: A_s}, this by no means guarantees that we have found the optimal analytical expression.

To simplify the user experience, this two-emulator solution lives ``under the
hood'' and by default \textcolor{orange}{will be} hidden behind an interface
which automatically queries the correct emulator given some user-input
cosmology.

To justify our decision and to quantify the improvement from this approach, we
have prepared \textcolor{orange}{some} plots in section~\ref{sec: 2emu_improvement}.

\textcolor{orange}{Conclude this section by saying that this integration of multiple emulators into one user-facing script can be further exploited with, for example, different emulators for different neutrino mass hierarchies.}

Besides its relevance to the neutrino suppression of the small-scale power
spectrum, $A_s$ appears to be an evolution parameter. Therefore, when the
emulator discussion branches into the massless- and massive-neutrino cases
(refer to section~\ref{sec: 2emu_intro} for the beginning and motivation of
this), the reader should keep in mind that the dimension of the parameter
space decreases in the massless-neutrino case by \textit{two}--$\omega_\nu$ 
obviously must vanish because we here assume $\omega_\nu=0$. However, $A_s$
also vanishes because, without massive neutrinos, $A_s$ reduces to an
evolution parameter and is therefore redundant with $\sigma_{12}$.

% What happens to A_s in the massless-neutrino case? Is it fixed at the
% default for model 0? I'm pretty sure it is!


\section{Unit LHCs}
\label{sec: lhc_flow_chart}

The \verb|pyDOE2| function that we will use to generate our LHCs
returns a result with entries that range from zero to unity. We will refer
to such LHCs as \textit{unit} LHCs.\footnote{We will later denote LHCs with 
the matrix notation $\matr{X}$ and collections of CAMB power spectra as
$\matr{Y}$ to
emphasize the roles of these data in the training and testing of the emulator. 
However, we will continue to use the term ``unit'' as introduced here, and 
the reader should keep in mind that a unit $\matr{X}$ therefore does not
indicate an identity matrix.} These unit LHCs will eventually be used 
to create a set of cosmological configurations which will act as the $X$ data 
set when we train our emulator.

In order to these LHC entries into cosmological parameters that we can input 
into CAMB, we will need to rescale them with our priors for each parameter.
However, instead of rescaling immediately, we will rescale later: 
\textcolor{orange}{The benefits of rescaling later are. This is useful not
just for changing priors but even for changing the way that we sample a
particular prior (for example, $\sigma_{12}^2$ sampling--although we did not
get it to work, the infrastructure is still there for whoever comes along in
the future.}
 
We describe the building of the unit LHC in section~\ref{sec: build_lhc}
and its rescaling (as well as the construction of the $Y$ data set) 
in~\ref{sec: generate_emu_data}.


\section{Default Priors}
\label{sec: default_priors}

% One reason this goes here: priors show up in almost every script,
% so there isn't a great cass-L section in which to put this.

As we will explain in section~\ref{sec: build_lhc}, our code generates unit 
Latin hypercube samples so that a single hypercube can be used to build 
several different emulators simply by using a different set of priors to 
rescale the cube.

CL is packaged with three pairs of prior files. Depending on which priors the 
user selects, the emulator will automatically become a massive-neutrino or 
massless-neutrino emulator. \textcolor{orange}{This needs to be linked to the 
convenience section, on how to add scenario files and what not. But this 
feature is not finished yet! We may have to punt it to “future work.”}

\textcolor{orange}{Table goes here.}
% Caption for one table: The ``classic'' priors represent a middle ground 
% between those of ``COMET'' and ``MEGA''.

In tables XXX-XXZ, we provide the specific upper and lower bounds for each of
the six cosmological parameters over which our massive-neutrino emulator is 
trained. All intervals are sampled uniformly, \textcolor{orange}{Although some
experimental code exists to interpret the same hypercube as a uniform sampling
in root or square space, for example}. At first, we used the highly ambitious 
(relative to modern uncertainty bars) ``MEGA'' set of priors
(table~\ref{tab: priors}),
whose values we decided based on the recent emulator papers from
\textcolor{green}{Spurio Mancini et al 2021. and the other paper, I can't
remember the name.} However, during the process of generating training data, 
we found that CAMB was not able to match all of the necessary $\sigma_{12}$ 
values. We refer to such a situation as an unsolvable cell, and will explain
why it happens in section~\ref{sec: generate_emu_data}, which covers the
application of evolution mapping principles within the CL code.

\textcolor{orange}{Talk a little more about the prior ranges, what are their
values in words??}

The default priors that we use correspond to those currently used by COMET. \textcolor{orange}{Two
justifications for this choice: unsolvable cells, but more than that, simplicity and accuracy
for the ``demo'' run.}

\section{Emulation over Uncertainties}

\textcolor{orange}{I'm nominating this section for deletion. I doubt I'll have
the time to put together a fully-fledged uncertainty emulator...}

For a further step of accuracy, we can add a third data set to our pipeline
and introduce a second layer of emulation.

Up to this point, our pipeline has included a training set and a testing set.
If we add a validation set, then we can train a second emulator over the
errors associated with the first emulator's performance on this validation
set.

Within the current (as of \textcolor{orange}{24.08.2023}) setup of
Cassandra-Linear, we typically generate two Latin hypercubes for each
emulator. The first represents our training set and an emulator cannot be
produced without it. The second one represents our validation set.

\textcolor{orange}{In a future
release of Cassandra-Linear, this set will be used to train an ``uncertainty''
emulator, which will train over the errors from the main emulator in order to
provide the user with an uncertainty estimate for any cosmology located within
the space of priors over which the main emulator was trained.} 

\textcolor{orange}{However, this functionality has not yet been implemented. 
Currently, we are 
using the validation hypercube more as a test hypercube: we compute the uncertainties at discrete points and examine these uncertainties (e.g. with
scatterplots and histograms) to assess the performance of the emulator.}
