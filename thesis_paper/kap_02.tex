\chapter{Adapting Evolution Mapping to Massive-neutrino Cosmologies}
\label{chap: A_s}

This chapter will motivate our two extensions of evolution mapping:
replacement of $\sigma_{12}$ with $\tilde{\sigma}_{12}$ and the inclusion 
of the additional parameter $A_s$ in the evolution mapping framework.
Because the focus of this thesis is regression rather than theory,
we will merely show that $A_s$ contains information relevant to the impact of
massive neutrinos. Then, the GPR should, in principle, be able to optimize its
use to predict power spectra.

% Maybe it would make more sense to have the large cass-L chapter focus on the
% creation of a massless-neutrino emulator and THEN a smaller chapter focusing
% on all the changes necessary for it to become a massive-neutrino emulator.
% BUT! As of 25-08-23, I'm running way behind on writing actual content for
% this thesis. I can't risk redoing all the section headers again. We're
% going to proceed under the current scheme and MAYBE allow ourselves to redo
% it shortly before submission.

\section{Characterizing the Power Spectrum Amplitude}

Consider the trivial exact relation

\begin{equation}
\label{eq: trivial}
P_\nu(k) = \de(k) \, P_0 (k)
,\end{equation}

where $\de(k) \equiv P_\nu (k) / P_0 (k)$, $P_\nu (k)$ is the power spectrum 
of the massive-neutrino cosmology in which we are interested, and $P_0(k)$ is 
the power spectrum of the corresponding matter-equivalent massless-neutrino
cosmology (MEMNeC). The MEMNeC is defined such that $\tilde{\omega}_c = 
\omega_c + \omega_\nu$ and $\tilde{\omega}_\nu = 0$. All symbols with tildes
indicate MEMNeC parameters, while symbols without tildes indicate parameters
in the original, massive-neutrino cosmology. Only the parameters $\omega_\nu$
and $\omega_c$ can differ between any cosmology and its MEMNeC.

% The following plots were generated with divergence_asymptotes.ipynb
\begin{figure}
    \centering
 	\includegraphics[width=0.67\textwidth]{de}
 	\caption[$\de(k)$ for the Aletheia models]{Here we have plotted
 		$\de(k)$ curves for a handful of cosmologies with the same shape
 		parameters but different evolution parameters.
 		The cosmologies are evaluated at different redshifts such that they 
 		have the same value of $\tilde{\sigma}_{12}$. Notice that all curves
 		converge to unity at the largest scales.}
 	\label{fig: model_ratios}
\end{figure}

To improve evolution mapping's effectiveness in massive-neutrino cosmologies,
we want to understand the behavior of $\de(k)$. In
figure~\ref{tilde_sigma_poc}, we show $\de(k)$ for a handful of cosmologies
with the same shape parameters but different evolution parameters. When
comparing different cosmologies at fixed $\tilde{\sigma}_{12}$, we find that
evolution mapping handles the large scales remarkably well, as $\de(k)$
invariably converges to unity. Consequently, we adapt the evolution mapping relation as follows:

\begin{equation}
\label{eq: evMapping_modded}
    P_L (k | z, \Theta_s, \Theta_e)
    =
    P_L (k | \Theta_s,
    		\tilde{\sigma}_{12} \left( z, \Theta_s, \Theta_e \right))
.\end{equation}

In other words, $\sigma_{12}$ is now replaced by $\tilde{\sigma}_{12}$, the
$\sigma_{12}$ value associated with the cosmology's MEMNeC.

\section{Characterizing Small-scale Deviations}

% Redo section title?

Next, we are interested in a way to extend the recipe also to smaller scales.
Consider again equation~\ref{eq: trivial}. We can approximate $\de(k)$ by 
fixing it to some $\de^*(k)$ that we calculate
once for each set of shape parameters.

\begin{equation}
\label{eq: MEMNeC_approx}
P_\nu(k) \approx \de^* (k) P_0 (k)
,\end{equation}

This limits the applicability of our 
emulator to cosmologies with small $\omega_\nu$, where we can treat the
interaction between massive neutrinos and evolution parameters as a 
perturbation on a baseline model.
We want to increase the accuracy in our $P_\nu(k)$
predictions by estimating the true $\de(k)$. Specifically, we are interested
in how the true $\de(k)$ deviates from $\de^*(k)$. Therefore, we pay special
attention to the quantity:

\begin{equation}
\label{eq: ee}
\ee (k) \equiv \frac{\de(k)}{\de^*(k)}
\end{equation}

To simplify the discussion, we will concentrate on the small-scale limit of 
this ratio $\el$

\begin{equation}
\el \equiv \lim_{k \rightarrow \infty} \ee(k)
\end{equation}

If we can estimate $\el$ accurately, then in principle our task 
is complete, because the slope toward smaller $k$ values is
even and analytically predictable. \textcolor{red}{Do you know of a relevant
paper to which I could refer the viewer? Or perhaps you would say that all of
the necessary information is already contained in FECS?}

\section{Proposed Fitting Function}
\label{sec: proposed_fit}
 
In order to explore this limit concretely, we will use the Aletheia models as
our test cases. This means that we can rewrite definition~\ref{eq: ee} as:

\begin{equation}
\label{eq: eei}
\el_i (k) \equiv \frac{\de_\text{model i}(k)}{\de_\text{model 0}(k)}
\end{equation}

To study the behavior of $\ee_i$, we developed the \verb|camb_interface.py|
function \\ \verb|model_ratios|. This function accepts a single snap
index\footnote{Remember that there are four snaps for each model, and snap 
four always corresponds to $z = 0$} and a set of power spectra nearly in the
format returned by
\verb|boltzmann_battery|, but without the $\omega_\nu$ 
layer.\footnote{For explanations of the
remaining function parameters, we refer the reader to the docstring} Tt
computes and plots all of the $\ee_i$ spectra in the input set. We show an
example output in figure~\ref{fig: model_ratios_demo}. This function
is recommended for any users seeking to improve the characterization of
$\ee_i$.

% The following plots were generated with divergence_asymptotes.ipynb
\begin{figure}
    \centering
 	\includegraphics[width=0.67\textwidth]{As_fit/predictor_success}
 	\caption[$\ee(k)$ for the Aletheia models]{Model ratios for the first
 		seven Aletheia models. The asymptotes predicted by our fit
 		(equation~\ref{eq: fit}) are plotted in sky blue as dashed
 		horizontal lines. The agreement appears sound in all cases.}
 	\label{fig: model_ratios}
\end{figure}

\textcolor{orange}{The above paragraph only describes the function in the
case ``massive=`x'''}

After experimenting with different functions and parameters, we find the
following fitting function:

\begin{equation}
\label{eq: fit}
\hat{\ee}_i = C \, \omega_\nu \, \ln \left( \frac{A_{s, i}}{A_{s, 0}} \right)
.\end{equation}

We demonstrate its predictions in figure~\ref{fig: model_ratios}.

We also list the errors for the Aletheia models in
table~\ref{tab: fit_errors_Aletheia}.

\begin{table}[ht!]
\centering
\begin{tabular}{l|l}
\hline
Model Index & {Percent Error} \\ \hline
1 & $-4.983 \cdot 10^{-3}$ \\
2 & $+2.503 \cdot 10^{-3}$ \\
3 & $-2.267 \cdot 10^{-3}$ \\
4 & $+2.238 \cdot 10^{-3}$ \\
5 & $+1.577 \cdot 10^{-3}$ \\
6 & $-3.137 \cdot 10^{-3}$ \\
6 & $-1.170 \cdot 10^{-2}$ \\
\end{tabular}
 \cprotect\caption[Fit Performance on Aletheia Models]{Errors, rounded to
 four significant figures, when predicting
 the $\el_i$ for various Aletheia models. As mentioned before, we skip model
 8, since the code is not setup to handle that exotic dark energy scenario.
 The reduced $\chi^2$ value across these models is $2.714 \cdot 10^{-9}$.}
 \label{tab: fit_errors_Aletheia}
\end{table}

\section{Testing the Fit}
\label{sec: fit_testing}

To verify that our approach has general applicability, we now test the
asymptotic fit on a much broader selection of models.

\textcolor{orange}{What if we tested this fitting function on an independent
set of models with a consistent but different set of shape parameters?}

First, we created the function \verb|get_random_cosmology| function, which
accepts an $\omega_\nu$ value and returns a cosmology with randomized
evolution parameters over the following ranges:~\ref{tab: fit_test_params}.
All of the shape parameters are still consistent with Aletheia model 0.

\begin{table}[ht!]
\centering
\begin{tabular}{l|l|l}
\hline
Parameter & Minimum Value & Maximum Value \\ \hline
$\omega_K$ & -0.05 & 0.05 \\
$\omega_\text{DE}$\footnotemark & 0.1 & 0.5 \\
$w_0$ & -2.0 & -0.5 \\
$w_a$ & -0.5 & 0.5 \\
$A_s$\footnotemark & $5.003 \cdot 10^{-10}$ & $1.484 \cdot 10^{-8}$  \\
\end{tabular}
 \cprotect\caption[Parameter Ranges for Random Test
 	Cosmologies]{Parameters which we vary for the purpose of testing the
 	generality of fit~\ref{eq: fit}, and their domains.
 	\textcolor{red}{Is it poor form to have a footnote that looks like an
 	exponent? How should I do it differently?}}
 \label{tab: fit_test_params}
\end{table}

\addtocounter{footnote}{-1}

\footnotetext{Unfortunately, $\omega_\text{DE}$ is not accepted as an
input by CAMB's \verb|set_cosmology|. To vary $\omega_\text{DE}$ in this way,
we fix $\omega_b$ and $\omega_c$ and select a value for $\omega_K$. Then, we
vary $h$--according to equation~\ref{eq: h_to_omega}, this should only
affect $\omega_\text{DE}$.}

% The following plots were generated with divergence_asymptotes.ipynb
\begin{figure}[ht!]
    \begin{subfigure}{0.45 \textwidth}
    \centering
 		\includegraphics[height=0.2\textheight]{As_fit/exotic_asymptotes}
 		\cprotect\caption{$\ee$ curves of the random models.}
 		\label{fig: random_battery}
    \end{subfigure}
    \begin{subfigure}{0.45 \textwidth}
    \centering
 		\includegraphics[height=0.2\textheight]{As_fit/exotic_histogram}
 		\caption{Errors associated with fit~\ref{eq: fit}.}
 		\label{fig: random_battery_errs}
    \end{subfigure}
        \centering
    \cprotect\caption[Random-cosmology Experiment]
    		{Results from a run of 100 cosmologies uniformly
    			random in the parameters listed in
    			table~\ref{tab: fit_test_params}, using the function
    			\verb|get_random_cosmology|. Since this test consists of
    			only one hundred models, the histogram cannot be interpreted as
    			comprehensive, but only as an example of typical errors in our
    			fit. Indeed, reruns of this experiment can lead to
    			significantly different spreads in the histogram.
    			The reduced $\chi^2$ value for this particular run of the
    			experiment is approximately $5.005 \cdot 10^{-7}$.
    		\textcolor{red}{Would it have been better if I just included the
    		right plot and no left plot?}
    		\textcolor{red}{UNACCEPTABLE! FONTS ARE VERY TINY!}}
    \label{fig: random_cosmology_experiment}
\end{figure}

\footnotetext{These values were rounded to three significant figures. Refer
to the code for greater precision.}

These cosmologies cover a wide range of $\el$ values, as shown in
figure~\ref{fig: random_battery}. The errors for the fits are shown in
figure~\ref{fig: random_battery_errs}.

To further test our solution, we also created the function \\
\verb|get_As_matched_random_cosmology|,
which randomizes the cosmologies in the same
way except for fixing $A_s$ at the value for model 0.
figure~\ref{fig: degenerate_battery}. The errors for the fits are shown in
figure~\ref{fig: degenerate_battery_errs}.

% The following plots were generated with divergence_asymptotes.ipynb
\begin{figure}[ht!]
    \begin{subfigure}{0.45 \textwidth}
    \centering
 		\includegraphics[height=0.25\textheight]{As_fit/degenerate_asymptotes}
 		\cprotect\caption{$\ee$ curves of the random models.}
 		\label{fig: degenerate_battery}
    \end{subfigure}
    \begin{subfigure}{0.45 \textwidth}
    \centering
 		\includegraphics[height=0.25\textheight]{As_fit/degenerate_histogram}
 		\caption{Errors associated with fit~\ref{eq: fit}.}
 		\label{fig: degenerate_battery_errs}
    \end{subfigure}
        \centering
    \cprotect\caption[$A_s$-degenerate Fit Test]
    		{Results from a run of 100 cosmologies uniformly
    			random in the first four parameters of
    			table~\ref{tab: fit_test_params}, using the function
    			\verb|get_random_As_matched_cosmology|. 
    			Since this test consists of
    			only one hundred models, the histogram cannot be interpreted as
    			comprehensive, but only as an example of typical errors in our
    			fit. Indeed, reruns of this experiment can lead to
    			significantly different spreads in the histogram.
    			The reduced $\chi^2$ value for this particular run of the
    			experiment is approximately $2.514 \cdot 10^{-8}$.
    			\textcolor{red}{Would it have been better if I just included the
    			right plot and no left plot?}
    			\textcolor{red}{UNACCEPTABLE! FONTS ARE VERY TINY!}}
    \label{fig: degenerate_cosmology_experiment}
\end{figure}

This third test demonstrates that the imperfections in our fit cannot be
attributed solely to the form of equation~\ref{eq: fit};  
since all of the cosmologies share $\omega_\nu$ and $A_s$ values, and since
the asymptotes were nevertheless not all identical, the
asymptote must have some dependence also on one of the other parameters that
we varied.  

However, the larger errors that we see in
figure~\ref{fig: random_battery_errs} suggests that our fit may only
approximately describe the relationship
between $A_s$, $\omega_\nu$, and the small-scale suppression of the power
spectrum. A symbolic regression investigation could prove highly effective at
resolving this ambiguity by efficiently searching out improved formulas. 
However, we do not consider this a promising avenue
for the continuation of this work (see section~\ref{sec: future_work} for our
recommendations); the error
is so low here that we do not believe the imperfect predictions to
significantly detract from the performance of the emulator. Besides this,
the purpose of our investigation here was only to roughly gauge the 
information content of $A_s$ with respect to the small-scale suppression of 
structure growth by massive neutrinos. The quality of our current fit, as
well as its limitations exposed by
figure~\ref{fig: degenerate_cosmology_experiment}, 

\section{Summary of Findings}

% I'm not sure if this section will survive, we'll see how big the conclusion
% here ends up being.

\textcolor{blue}{Now you also have to include a summary of the finding on
$\sigma_{12}$. Remember to include in the new section 1.1 the ratio
plots, the first ones Ariel ever asked you to make. Those prove the 
effectiveness of the trick.}

In this chapter, we have illustrated how the scalar mode amplitude $A_s$
contains information of great relevance to the application of evolution
mapping to the emulation of massive-neutrino power spectra. As a result of 
these demonstrations, we know to construct 
our massive-neutrino emulator emulator over the six cosmological
parameters $\omega_b$, $\omega_c$, $n_s$, $\tilde{\sigma}_{12}$, $A_s$, and
$\omega_\nu$. In other words, to predict a power
spectrum, our emulator will accept as an input a vector describing
these six parameters. Conceptually, adding $A_s$ to the set of
emulation parameters allows us to treat $\omega_\nu$ like a shape
parameter.

Expansion of the parameter space represents the single most 
important novel step in this work. The remaining chapters will
focus on the integration of this modified evolution mapping framework
into a beginner-friendly emulation code.
