\chapter{CAMB, Initial Setup}

%%%%%%%%%%%%%%%%%%%%%%%%%%%%%%%%%%
%%  Beispiel fuer eine Tabelle  %%
%%%%%%%%%%%%%%%%%%%%%%%%%%%%%%%%%%

\begin{table}[htb]
\centering
\begin{tabular}{l|l}
Erste Spalte & Zweite Spalte \\ \hline
Eintrag & Eintrag
\end{tabular}
 \caption[Kurzform f"ur das Tabellenverzeichnis]{Dies ist die Erkl"arung zur Tabelle.}
\end{table}

CAMB is a Fortran code with a Python wrapper\footnote{
\url{https://github.com/cmbant/CAMB}
}which we will be using for the
entirety of this project.

To introduce the reader to the scope of CAMB, we will now introduce
some basic simulated power spectra along with a summary of the dynamic
parameters which will be of greatest interest to us.

I hope to, in painstaking detail, cover many of the lines of the code that I
have written to interface with CAMB. I will include plots to indicate, at
every step, what incorrect settings cause the power spectrum to look like (or,
for subtler errors, what the error curves looked like compared to Ariel's
results, which I treated as a sort of ``ground truth''). This should also be a
good example to flex my physics interpretation skills: why does this incorrect
setting produce this undesired pattern?

You might think that this is sort of an inappropriate section for a master's thesis (especially since I have in mind that this be a lengthy section), but I would like to include it unless you feel very strongly. After all, I spent several months of the project debugging at least ten different ways that slight and major errors in the various settings led to irreconcilable results.

For example, one parameter that tripped me up for a while: neutrino mass hierarchy: the options are degenerate, normal, and inverted. The CAMB documentation annotates this parameter as ``(1 or 2 eigenstate approximation),'' but this is somewhat unclear. Is the degenerate hierarchy the single mass eigenstate approximation? Do both normal and inverted hierarchies involve two eigenstates?

%In figure \ref{fig: spectrum_type}, we can see that requesting of the wrong
%power spectrum type can in some low-$\omega_\nu$ cases yields errors so low
%that we might accidentally overlook them. This error pattern is easily
%recognizable and is a consequence of the definition of the power spectrum: the
%Fourier transform  of the two-point correlation function. ...Okay, I'm still thinking about this. I don't understand %yet, but I'll be sure to ask you if I'm still struggling about it.

Another paragraph I want to have in this section: stress the part of the evolution mapping introduction, that the $\sigma_{12}$ value we're using to describe the model is actually the $\sigma_{12}$ value of the model's MEMNeC! This is so important and confusing that maybe I'll even recapitulate again later in the section on the generate\_emu\_data.script.

%%% New stuff

Beware the neutrino settings. The effective number of massive neutrinos is about 3.027

%! This is weird, and perhaps inappropriate for a thesis document.

To test this setup, we compare our results with those of Ariel S\'{a}nchez and
Andrea Pezzotta for the first seven Aletheia cosmologies and four physical
densities in neutrinos, for a total of 28 models. The errors are miniscule and
recorded in figure XXY. After verifying the accuracy of our code in this way,
we proceed to experiment with the power spectra in order to explore solutions
to the evolution mapping problem.