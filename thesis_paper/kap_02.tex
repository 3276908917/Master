\chapter{CAMB, Initial Setup}

\section{Introduction to CAMB}

CAMB is a Fortran code with a Python wrapper\footnote{
\url{https://github.com/cmbant/CAMB}
} which we will be using for the
entirety of this project.

To introduce the reader to the scope of CAMB, we will now introduce
some basic simulated power spectra along with a summary of the dynamic
parameters which will be of greatest interest to us.

\section{Configuring CAMB for this Project}

\begin{comment}
\textcolor{blue}{
I hope to, in painstaking detail, cover many of the lines of the code that I
have written to interface with CAMB. I will include plots to indicate, at
every step, what incorrect settings cause the power spectrum to look like (or,
for subtler errors, what the error curves looked like compared to Ariel's
results, which I treated as a sort of ``ground truth''). This should also be a
good example to flex my physics interpretation skills: why does this incorrect
setting produce this undesired pattern?}

\textcolor{blue}{You might think that this is sort of an inappropriate 
section
for a master's thesis (especially since I have in mind that this be a lengthy 
section), but I would like to include it unless you feel very strongly. After 
all, I spent several months of the project debugging at least ten different 
ways that slight and major errors in the various settings led to 
irreconcilable results.}
\end{comment}

\textcolor{orange}{Ariel recommends
just talking about the correct lines, don't talk about what happens when
they're wrong.}

%For example, one parameter that tripped me up for a while:
The impact of some parameters can be quite subtle, especially (we imagine)
for users unfamiliar with the nuances of neutrino physics.

Consider the neutrino mass 
hierarchy: the options are degenerate, normal, and inverted. The CAMB 
documentation annotates this parameter as ``(1 or 2 eigenstate 
approximation),'' but this is somewhat unclear. Is the degenerate hierarchy 
the single mass eigenstate approximation? Do both normal and inverted 
hierarchies involve two eigenstates? Besides, even if this description were
literally accurate, why does use of the normal mass hierarchy lead to
incorrect results?

% C'mon, Ariel's GOT to let me have this plot. It's a big open question!

%In figure \ref{fig: spectrum_type}, we can see that requesting of the wrong
%power spectrum type can in some low-$\omega_\nu$ cases yields errors so low
%that we might accidentally overlook them. This error pattern is easily
%recognizable and is a consequence of the definition of the power spectrum: the
%Fourier transform  of the two-point correlation function. ...Okay, I'm still thinking about this. I don't understand %yet, but I'll be sure to ask you if I'm still struggling about it.

Another paragraph I want to have in this section: stress the part of the evolution mapping introduction, that the $\sigma_{12}$ value we're using to describe the model is actually the $\sigma_{12}$ value of the model's MEMNeC! This is so important and confusing that maybe I'll even recapitulate again later in the section on the generate\_emu\_data.script.

%s Now discuss individual settings

In principle, the accuracy boost should increase the variance in our models but should not affect the bias. The line is

\verb|pars.Accuracy.AccuracyBoost = 3|

\textcolor{orange}{But what does it do? The CLASS citation may have some
useful information!}

\verb|pars.NonLinear = camb.model.NonLinear_none|

% Why is CAMB set up this way anyway? Why do I have to put in a model object 
% instead of simply flipping a Boolean? {probably not a relevant question for
% this thesis.}

\verb|pars.WantCls = False|

``\verb|WantCls| – (boolean) Calculate C\_L.'' \textcolor{orange}{This entry 
of
the CAMB documentation is NOT helpful. Looks like we’ll have to do some
detective work.}

\verb|pars.WantScalars = False|

``\verb|WantScalars| – (boolean) Calculates scalar modes.'' We should briefly 
talk about what they are, then say ``we are not interested in these, let's 
save time by skipping the computation.''

\verb|pars.Want_CMB = False|

``Calculate the temperature and polarization power spectra.'' We're not 
interested in these quantities; in order to apply computational resources more 
efficiently, we should avoid these calculations and therefore set this to 
\verb|False|.

\verb|pars.DoLensing = False|

``Include CMB lensing'' but \textcolor{red}{why don't we care about this? Does
this not affect the power spectrum results in any way that we can appreciate
within the confines of the MCMC experiment that we're testing here?}

\verb|pars.YHe = 0.24|

\verb|pars.Accuracy.lAccuracyBoost = 3|

\verb|pars.Accuracy.AccuratePolarization = False|

%s Small things

Beware the neutrino settings. The effective number of massive neutrinos is about 3.027

\textcolor{orange}{We'll also want to justify why we're using k ranges out to
such tiny scales. It's true that our surveys can't probe these scales, but
this work is purely computational. In its application, the high k values can
always be disregarded.}

\section{Convenience Functions}

% camb\_interface.py

\textcolor{blue}{The purpose of this section is to anticipate the Python
package Cassandra-Linear. The settings we describe in the previous section
still technically correspond to lines of code of course, but here we tie these
individually-treated lines into bigger functions that users can quickly use to
get results from CAMB!}

\section{Verifying Our Settings}

%! This is weird, and perhaps inappropriate for a thesis document.

To test this setup, we compare our results with those of Ariel S\'{a}nchez 
(within the native Fortran framework of CAMB) and Andrea Pezzotta for the
first seven Aletheia cosmologies and four $\omega_\nu$ physical densities in
neutrinos (\textcolor{green}{approximately} 0, 0.0006, 0.002, 0.006.), for a 
total of 28
models. The errors are miniscule and recorded in figure XXY. After verifying
the accuracy of our code in this way, we proceed to experiment with the power
spectra in order to explore solutions to the evolution mapping problem.

% Aletheia Model 8 has not been integrated into our code suite yet. We can’t handle its DE weirdness. Instead of bringing up and then discarding model 8, we shouldn’t mention it at all.

