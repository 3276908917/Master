\chapter{Results and Analysis}

Unless otherwise stated, all emulated results shown were generated by a standard two-emulator 5000 sample each setup, where each training point consists of 300-k training spectra.

\section{Quantifying the Performance of the Emulator}

Before I even show any results for this emulator, I would like to motivate the challenge of gauging the accuracy and reliability of the emulator. Since we have to quantify performance over a hypervolume of parameter space, there is no concept of, for example, a chi-squared test that can applied here. So I am trying to pre-empt objections like those raised by Stella.

I would like to include a brief review of the relevant literature to explain the kinds of performance metrics that are most commonly applied in emulator papers (as I remember, largely percent-level bounds when comparing against ground truth data sets). Then, I would like to break down Ariel's arguments for why absolute error is of greater consequence to us than relative errors.

For the purpose of this paper, we use the weak and approximate performance metric of simply generating a large number of additional CAMB spectra and comparing them to the predictions of the emulator.

\section{Percent versus Absolute Errors on Random Cosmologies}

This will be a fairly short section, basically just showing the plot of 5000 error curves in these two ways. I may focus in on different k-ranges, but the error curves are currently quite flat, so I don't think that would be a good use of space.

Furthermore, I will try to select a couple of cosmologies out of the 5000 (maybe a low-error case and a high-error case) as examples of the performance on just one at a time. But I'm not sure how insightful that will be, I don't know if that will tell the reader a whole lot.

\section{Performance in Different Parameters}

Here, I will either include color plots or, as Dante suggests, monochrome plots with error as one axis and parameter value as the other (i.e. $k$ fixed). Then, if I haven't tightened the $\sigma_{12}$ performance by the time of submission, I can talk about how this is the most promising avenue for refinement of the emulator. In any case, I plan to spend some time talking about \textit{why} parameter x is the current biggest problem for the emulator. 

\section{Improvement from Two-emulator Solution}
\label{sec: 2emu_improvement}

\textcolor{blue}{This will be an extremely short section with some error
plots of the massive-neutrino cosmology evaluating massless-neutrino
cosmologies.}