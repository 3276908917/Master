\chapter{Expansion of the Parameter Space}

% Section on messing around

Here, I will talk about the playing-around that we did, to discover that
$A_s$ assisted in the prediction of the suppression due to massive neutrinos.
Specifically, we just need to add $A_s$ to the set of cosmological parameters
over which we train our GPR and then we can treat $\omega_\nu$ like a shape
parameter.

In producing asymptote plots, we find that our predictions line up quite well,
but not perfectly. This may indicate that our parametrization is insufficient.
In other words, we may need additional cosmological parameters besides the
$A_s$ and $\omega_\nu$ and in order to fully characterize the impact of
massive neutrinos on the power spectrum.

However, the imperfect performance of our predictions may also simply
indicate that the final form of our asymptote predictions (equation XXY) is
only an approximation of some true (or at least more accurate) relationship
between $A_s$, $\omega_\nu$, and the small-scale suppression of the power
spectrum. A symbolic regression investigation could prove highly effective at
resolving this ambiguity by efficiently searching out improved formulas. 
However, we do not consider this a promising avenue
for the continuation of this work (see section~\ref{future_work} for our
recommendations), because the error on our predictions is so low. The error
is so low here that we do not believe the imperfect predictions to
significantly detract from the performance of the emulator.

% Maybe it would make more sense to have the large cass-L chapter focus on the
% creation of a massless-neutrino emulator and THEN a smaller chapter focusing
% on all the changes necessary for it to become a massive-neutrino emulator.
% BUT! As of 25-08-23, I'm running way behind on writing actual content for
% this thesis. I can't risk redoing all the section headers again. We're
% going to proceed under the current scheme and MAYBE allow ourselves to redo
% it shortly before submission.

So, our massive-neutrino emulator will be trained over six cosmological
parameters: $\omega_b$, $\omega_c$, $n_s$, $\sigma_{12}$, $A_s$, and
$\omega_\nu$. In other words, to predict a power
spectrum $\hat{y}$, the emulator will accept as an input $x$ these six
parameters.

Besides its relevance to the neutrino suppression of the small-scale power
spectrum, $A_s$ appears to be an evolution parameter. Therefore, when the
emulator discussion branches into the massless- and massive-neutrino cases
(refer to section~\ref{sec: 2emu_intro} for the beginning and motivation of
this), the reader should keep in mind that the dimension of the parameter
space decreases in the massless-neutrino case by \textit{two}--$\omega_\nu$ 
obviously must vanish because we here assume $\omega_\nu=0$. However, $A_s$
also vanishes because, without massive neutrinos, $A_s$ reduces to an
evolution parameter and is therefore redundant with $\sigma_{12}$.

% What happens to A_s in the massless-neutrino case? Is it fixed at the
% default for model 0? I'm pretty sure it is!

\textcolor{orange}{Remember to explain \textit{why} the prediction of these 
asymptotes means that $A_s$ will capture most of the unruly behavior of 
neutrinos. What theory motivates such a judgment? Or are we only guessing?}

\textcolor{orange}{This section, like the ``Convenience Functions'' section
from chapter 2, should anticipate Cassandra-Linear: we should talk about the
code that we have written in order to explore these ratios, which one can find
in} \verb|camb_interface.py|.
