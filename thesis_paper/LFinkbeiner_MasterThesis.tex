%%%%%%%%%%%%%%%%%%%%%%%%%%%%%%%%%%%%%%%%%
%%            LMU-Vorlage              %%
%%                                     %%
%%         zur Erstellung einer        %%
%%   Dissertation mit pdflatex/latex   %%
%%                                     %%
%%  (2002) Robert Dahlke               %%
%%         & Sigmund Stintzing         %%
%%%%%%%%%%%%%%%%%%%%%%%%%%%%%%%%%%%%%%%%%

\documentclass[12pt]{book}


%%%%%%%%%%%%%%%%%%%%%%%%%%%%
%%   Zusaetzliche Pakete  %%
%%%%%%%%%%%%%%%%%%%%%%%%%%%%

\usepackage{a4wide}
\usepackage{fancyhdr}
\usepackage{graphicx}
\usepackage{german}
\usepackage[bookmarks]{hyperref}

% Additional packages just for Lukas' thesis
\usepackage{xcolor, bbold, graphics}
\usepackage{amsmath, verbatim, bm}
\usepackage{ulem}

%\IfFileExists{biblatex.sty} {
%    \usepackage[style=authoryear]{biblatex}
%    \addbibresource{literatur.bib}
%}

% This code formatting sucks but I don't want to output extra spaces.
\newcommand{\cbib}[1]
{\IfFileExists{biblatex.sty}
{\cite{#1}}
{[citation ``#1'' cannot be linked in the current environment]}}

%%%%%%%%%%%%%%%%%%%%%%%%%%%%%%
%% Definition der Kopfzeile %%
%%%%%%%%%%%%%%%%%%%%%%%%%%%%%%

\pagestyle{fancyplain}
\renewcommand{\chaptermark}[1]%
         {\markboth{\thechapter.\ #1}{}}
\renewcommand{\sectionmark}[1]%
         {\markright{\thesection\ #1}}
\lhead[\fancyplain{}{\bfseries\thepage}]%
    {\fancyplain{}{\bfseries\rightmark}}
\rhead[\fancyplain{}{\bfseries\leftmark}]%
    {\fancyplain{}{\bfseries\thepage}}
\cfoot{}


%%%%%%%%%%%%%%%%%%%%%%%%%%%%%%%%%%%%%%%%%%%%%%%%%%%%%
%%  Definition des Deckblattes und der Titelseite  %%
%%%%%%%%%%%%%%%%%%%%%%%%%%%%%%%%%%%%%%%%%%%%%%%%%%%%%

\newcommand{\LMUTitle}[9]{
  \thispagestyle{empty}
  \vspace*{\stretch{1}}
  {\parindent0cm
   \rule{\linewidth}{.7ex}}
  \begin{flushright}

    \vspace*{\stretch{1}}
    \sffamily\bfseries\Huge
    #1\\
    \vspace*{\stretch{1}}
    \sffamily\bfseries\large
    #2
    \vspace*{\stretch{1}}
  \end{flushright}
  \rule{\linewidth}{.7ex}
  \vspace*{\stretch{5}}
  \begin{center}
    \includegraphics[width=2in]{siegel}
  \end{center}
  \vspace*{\stretch{1}}
  \begin{center}\sffamily\LARGE{#5}\end{center}
  \newpage
  \thispagestyle{empty}

  \cleardoublepage
  \thispagestyle{empty}

  \vspace*{\stretch{1}}
  {\parindent0cm
  \rule{\linewidth}{.7ex}}
  \begin{flushright}
    \vspace*{\stretch{1}}
    \sffamily\bfseries\Huge
    #1\\
    \vspace*{\stretch{1}}
    \sffamily\bfseries\large
    #2
    \vspace*{\stretch{1}}
  \end{flushright}
  \rule{\linewidth}{.7ex}

  \vspace*{\stretch{3}}
  \begin{center}
    \Large Dissertation\\
    \Large an der #4\\
    \Large der Ludwig--Maximilians--Universit"at\\
    \Large M\"unchen\\
    \vspace*{\stretch{1}}
    \Large vorgelegt von\\
    \Large #2\\
    \Large aus #3\\
    \vspace*{\stretch{2}}
    \Large M\"unchen, den #6
  \end{center}

  \newpage
  \thispagestyle{empty}

  \vspace*{\stretch{1}}

  \begin{flushleft}
    \large Erstgutachter:  #7 \\[1mm]
    \large Zweitgutachter: #8 \\[1mm]
    \large Tag der m\"undlichen Pr\"ufung: #9\\
  \end{flushleft}

  \cleardoublepage
}




%%%%%%%%%%%%%%%%%%%%%%%%%%%%
%%  Beginn des Dokuments  %%
%%%%%%%%%%%%%%%%%%%%%%%%%%%%

\begin{document}


  \frontmatter


  \LMUTitle
      {Adapting Evolution-Mapping Emulators for Massive-Neutrino Cosmologies}               % Titel der Arbeit
      {Lukas Finkbeiner}                       % Vor- und Nachname des Autors
      {Boston}                             % Geburtsort des Autors
      {Fakultät für Physik}                         % Name der Fakultaet
      {M"unchen 2022-23}                          % Ort und Jahr der Erstellung
      {2. Oktober 2023}                            % Tag der Abgabe
      {Erstgutachter}                          % Name des Erstgutachters
      {Zweitgutachter}                         % Name des Zweitgutachters
      {Pr"ufungsdatum}                         % Datum der muendlichen Pruefung


  \tableofcontents
  \markboth{Inhaltsverzeichnis}{Inhaltsverzeichnis}


  \listoffigures
  \markboth{Abbildungsverzeichnis}{Abbildungsverzeichnis}


  \listoftables
  \markboth{Tabellenverzeichnis}{Tabellenverzeichnis}
  \cleardoublepage


  \markboth{Zusammenfassung}{Zusammenfassung}
  \addcontentsline{toc}{chapter}{\protect Zusammenfassung}


\chapter*{Zusammenfassung}

The color code in this document: \textcolor{blue}{Explanations to the
proofreader; we'll simply cut these out shortly before submission.}
\textcolor{orange}{Notes to myself--I already know what I would like to do to
enhance these areas.} \textcolor{green}{I need to add citations / 
justification for the claims here, either by generating more plots or by going
back and looking through the papers I used to get started with this project.}
\textcolor{red}{I have some confusion about this area and I would appreciate
feedback from the proofreader.}

\textcolor{red}{I feel like the extreme simplifications in the next paragraph
would be helpful for someone with no idea about this field, but what do you
think? Maybe they would be more appropriate for easing the reader into
particular sections on these topics?}

``The goal of this project is to produce an emulator of the linear-theory
power spectrum based on evolution mapping'' (A. G. S\'{a}nchez, private
communication). An emulator can be thought of as a multi-dimensional function
produced by interpolation across many training points $(x, y)$. The power
spectrum can be thought of as a statistical description of the ``clumpiness''
of matter in the Universe. Evolution mapping is a technique for simplifying
the parameter space by identifying many cosmological parameters as basically
the same in their impact on the power spectrum.

The evolution mapping scheme of \cbib{San21} has already achieved success in
emulators of the nonlinear power spectrum. However, evolution mapping has 

his work seeks to extend the
evolution mapping scheme to massive-neutrino cosmologies by applying a
correction
factor to results from emulators built on massless-neutrino simulations. We
find that the scalar mode amplitude $A_s$ can be used to quantify the
suppression of structure growth due to massive neutrinos. Consequently, by
including this parameter, we can successfully train over the physical density
of the universe in massive neutrinos. We introduce a new emulation code,
Cassandra-Linear, which combines this expanded parameter space with evolution
mapping. We include various error statistics and show that the emulator
performs roughly at the level of error associated with CAMB itself
\textcolor{green}{Do I have a citation for the error associated with CAMB?}


  \mainmatter\setcounter{page}{1}
  \chapter{Introduction, Theory, and Background}

\textcolor{blue}{I have a lot of different important concepts that I need to 
get through, so I can easily imagine this becoming a relatively long 
introduction compared to other master's theses.}

A primary goal of cosmology is to specify, as narrowly as possible, the 
parameters which define our Universe. These include, for example, the overall 
curvature of the Universe as well as its cold dark matter (CDM) content. These
parameters determine the full evolution of the Universe after the inflationary
period (whose beginnings were thought to be non-deterministic)
\cbib{Caravano}. For example, depending on
the makeup of the Universe--how much of its total energy budget exists in the
form of each `ingredient' (cold dark matter, radiation, etc.)--the Universe
can have a finite lifetime. When the proportion of matter is high enough,
gravity will cause the Universe to collapse again on itself. By contrast, if
the proportion of dark energy is high enough, the Universe will continue to
accelerate in its expansion forever.

Cosmology has rapidly evolved into a high-precision science. For example, with
COBE (1989-1993) \cbib{COBE} followed by WMAP (2001-2010) \cbib{WMAP} followed 
by Planck (2009-2013) \cbib{Planck}, the uncertainties on several cosmological 
parameters have been tightened significantly.
\textcolor{red}{Should I try to offer concrete examples of how these missions
increased the precision on our parameter values? I feel like that might not be
a good use of space in the intro.}

Ultimately, the goal of this work is to speed up the kinds of statistical
analyses which are necessary to tighten these uncertainties.
These analyses compare our cosmological observations to what we
would expect to see if we solved the equations of cosmological evolution with
different values for parameters.

\section{Brief Glossary of Our Cosmological Parameters}
\label{sec: param_glossary}

\textcolor{blue}{We need to explain what the different parameters mean! The
big omega terms are likely to be more familiar to readers, we can start with 
those.}

We will get more specific about terms like ``cosmological observations''
and ``what we would expect to see'' in the next section (\ref{sec: Pk_intro}, 
on the matter power spectrum). First, we will briefly introduce concrete
examples of cosmological parameters in which we are interested.

In this work, we will concentrate on different parameters to different 
extents. By the end of this thesis, we will have built and tested two main
emulators, each of which is trained over a different parameter
space.\footnote{Emulators will be described in greater detail in
section~\ref{sec: emulation_intro}. For now, the extremely simple definition 
from the summary will suffice.}
Each emulator accepts as $\matr{X}$ sets of tuples of values for cosmological 
parameters and predicts as $\hat{\matr{Y}}$ the power spectra to which these
tuples correspond. 

%s Introduce the parameter names, THEN talk about what they mean

\textcolor{orange}{This paragraph needs to be rewritten. It's way too messy to 
talk about parameters in order of importance!}

A secondary, ``support'' emulator has been trained over four parameters:
$\omega_b$, $\omega_\text{CDM}$, $n_s$, and $\sigma_{12}$.
The primary emulator for this work has been trained over two additional
parameters: $A_s$ and $\omega_\nu$. Later in this section, we will also 
introduce the terms $h$ and $z$, which are not subject to emulation but which 
will be important to the generation of training data for the emulator (see 
section~\ref{sec: train_emu}. Finally, we will very briefly define terms such 
as $\Omega_k$, $\omega_\text{DE}$, $w_0$, $w_a$ since they make only minor 
appearances in this work.

%s First: elucidate the density parameters

The various $\omega_i$ and $\Omega_i$ terms are all related, and describe the
energy density of the Universe in different energy species.

%s First talk about what \omega and \Omega mean, then talk about what the
%s subscripts mean

%s What do the subscripts mean?

\begin{table}[htb]
\centering
\begin{tabular}{l|l}
\hline
Symbol & Energy Species \\ \hline
$\gamma$ & Relativistic (i.e. radiation, photons) \\
$B$ or $b$ & Baryons \\
$C$, $c$, or CDM & Cold dark matter \\
$\nu$ & Neutrinos \\
$M$ & Matter ($b$, CDM, and $\nu$ together) \\
$K$, $k$, or $\kappa$ & Curvature \\ \hline
\end{tabular}
 \caption[Energy species symbols]{The various energy species into which the 
 	contents of the Universe are categorized, and their conventional symbols.}
 \label{tab: species_symbols}
\end{table}

Table~\ref{tab: species_symbols} gives the various subscripts and the energy species to which
they correspond. We note that $\omega_\gamma$ does not appear in this thesis;
while radiation is critically important in the early stages of the
Universe, it rapidly becomes inconsequential
and radiation-matter equality occurs only 55,000 years after the
beginning of the Universe (\cbib{CO}, page 1194).

While baryons and cold dark matter (CDM) are both pressureless\footnote{This 
is of
course an approximation, but it is a very good one. While a hot baryonic gas 
certainly exerts a pressure, this pressure is negligible in comparison to the energy density of the gas. \textcolor{red}{Where are we getting a unitless
pressure term such that we can compare it in this way to energy density?
Otherwise, terms of different units cannot be compared. I understand that we
can get a unitless energy density by making it relative. Is there a similar
procedure for pressure terms?}}, CDM appears to interact only gravitationally
with other matter. This lack of collisional and radiative self-interaction
prevents CDM clouds from collapsing in the same way that, for example,
baryons can collapse to form stars and planets. Because of the significantly
different behaviors of these two species of matter, we treat them separately
even in cosmological contexts.

{segue from power spectrum discussion?} There is another facet to densities that will be conceptually important to this work: densities $\rho_i$ of different energy species. Similarly to before, it is common to define a related quantity $\Omega_i$ that is normalized, although here the normalization is enforce

\begin{equation}
\sum_i \Omega_i = 1
\end{equation}

IT’s MORE COMPLICATED THAN THAT—TALK ABOUT FLATNESS OF UNIVERSE and critical densities. These $\Omega_i$ parameters are referred to as “fractional energy densities.” The constant of normalization is simply $h^2$. $h$ is known as
the dimensionless Hubble parameter and is merely a different 

\begin{equation}
h = H_0 \, \frac{1}{100 }
\end{equation}

For example, the Planck best-fit value for $H_0$,
$67 \, \frac{\mathrm{km} / \mathrm{s}}{\mathrm{Mpc}}$,
corresponds to the $h$ value $0.67$. 

Unfortunately, this normalization has its disadvantages. WHAT ARE THEY. In particular, in the context of parameter inference, it is inconvenient to try to constrain a term consisting of multiple uncertain parameters. In this case, $h$ is itself a parameter that we would like to constrain, so it would be better to keep it separate. This situation is made more drastic due to the Hubble tension, which leaves the value of $h$ significantly unclear compared to other parameters.

throughout this paper we will refrain from using the conventional
fractional density parameters $\Omega_i$ and use instead the physical density
parameters $\omega_i$.

%s omega_k

According to Andrea P., $\omega_k$ is an evolution parameter only in a narrow 
band around zero--specifically, over roughly the range [-0.05, 0.05].
\textcolor{red}{By how much does evolution mapping fail here? And, if 
$\omega_k$ is truly a shape parameter, why does Ariel's FECS call $\omega_k$
an evolution parameter?} We do not emulate over $\omega_k$ but assume take
it to be zero for all cells. However, as we explain in
section~\ref{sec: ev_mapping_intro}, this does not limit the applicability of
our emulators. The use of nonzero $\omega_k$ values will simply entail a
relabeling of the emulated power spectra. \textcolor{orange}{Then again, I'm
holding $\sigma_{12}$ fixed, so does it change anything at all?}

%s n_s

The parameter $n_s$ is called the spectral index and describes the power laws
to which the power spectrum may be fit. We will delay its explanation to
section~\ref{sec: Pk_intro}, where we introduce the power spectrum.

%s sigma12 versus sigma8. This should come last as it is by far the most
%s challenging of the parameters to explain.

\textcolor{orange}{We should also have the $\sigma_8$ section here! This 
section will include an extremely brief summary of Ariel's paper motivating 
the use of $\sigma_{12}$ instead of $\sigma_8$.}

When comparing $\sigma_8$ results from different analyses, for example, the significance of the parameter itself is easily lost. Whenever two analyses differ, even slightly, in their values for $h$, the meaning of sigma 8 changes!

Since $h$ is already its own parameter, the conventional $\sigma_8$ parameter 
is truly a mixture of two parameters. This presents a host of misleading 
results and statistical ambiguities (\cbib{San20}) which are outside of the 
scope of this work (\textcolor{orange}{Nevertheless I would like to summarize 
a couple of key arguments}) but which prompt us to abandon $\sigma_8$.

In the end, however, the argument in favor of $\sigma_{12}$ most relevant to
this paper is the evolution mapping argument. We can greatly simplify the
parameter space of emulators with the help of $\sigma_{12}$, and $\sigma_8$
is simply incapable of exploiting the parameter degeneracies to the same
extent.
 

\section{The Matter Power Spectrum}
\label{sec: Pk_intro}

%%%%%%%%%%%%%%%%%%%%%%%%%%%%%%
%%  Einbinden einer Grafik  %%
%%%%%%%%%%%%%%%%%%%%%%%%%%%%%%

\begin{figure}[htb]
  \centering
  \includegraphics[scale=0.5]{siegel}
  \caption[Kurzform f"ur das Abbildungsverzeichnis]{Dies ist die Erkl"arung zum Bild.}
\end{figure}

How can we describe the Universe in such 
a way as to allow quantitative definitions of these phrases? As a starting 
point, we can imagine quantifying the energy density $\rho(\bm{x})$ of the
Universe at any one point $\bm{x}$ in three-dimensional space.

Let us return to the idea that cosmological parameters determine the the 
evolution of the Universe.
% In what ways can we quantify the Universe, in order to be able to describe
% its evolution? We can simply consider the time dependence of the density.

We can define a similar quantity, a relative density:

\begin{equation}
\delta(\bm{x}) = \frac{\rho(\bm{x}) - \bar{\rho}}{\bar{\rho}}
\end{equation}

which we refer to as the ``matter density contrast field.'' $\bar{\rho}$
represents the average energy density of the entire Universe.

$\delta(\bm{x})$ easier than $\rho(\bm{x})$ to work with since it
is unitless and especially because it is normalized--the integral of the
matter density contrast field, taken over the entire Universe, is unity. One particularly popular metric is the cosmic matter density 
contrast field. evolution of the Universe? The various constituents of the Universe 

Next, I want to talk about one way of describing the matter density contrast 
field: the power spectrum. The power spectrum can be probed in many different 
ways, and its precise shape and amplitude can tell us about several of these 
cosmological parameters.

Actually, the power spectra we are discussing in this thesis are linear-theory
power spectra of non-neutrino matter. \textcolor{blue}{But anyway, here I will cover some of 
the tried-and-true basic explanations of what the power spectrum is and why it 
is interesting for the question of parameter inference. I also want to 
discuss: why do we care about the linear-theory power spectrum? Why not jump 
straight to the nonlinear case?}

\textcolor{red}{The argument on the exponential should be a dot product, 
right?}

\begin{equation}
\tilde{\delta} (\bm{k}) = \int d^3 x \delta(\bm{x}) \exp(-i \bm{k} \bm{x})
\end{equation}

\begin{equation}
\langle \tilde{\delta} (\bm{k}_1) \tilde{\delta} (\bm{k}_2)^* \rangle
=
(2 \pi)^3 \delta_D^{(3)} (\bm{k}_2 \ \bm{k}_1) P(\bm{k})
\end{equation}

%s Why do we care?

The power spectrum is of great importance to parameter inference because of
its unique dependence on certain parameters.
% Here: refer to specific P(k) equations where omega_b and n_s showed up,
% turnover scale.
In section~\ref{sec: boltzmann_intro}, we will include several figures 
illustrating more clearly what we mean by this.

%s What is the BAO, what is the turnover scale

\textcolor{orange}{WE SHOULD DEFINITELY incorporate some of ARIEL'S DISCUSSION 
OF THE BAO 
(``WHY ARE THERE
WIGGLES IN THE POWER SPECTRUM'') AND WE SHOULD DO ENOUGH OF THE MODE / HORIZON
DISCUSSION TO ELUCIDATE THIS CONCEPT OF A TURNOVER SCALE!!!}
\cbib{FECS}

%s Everyone else uses h units. Why don't we?

\textcolor{blue}{This section will also reiterate some parts of 
section~\ref{sec: param_glossary} and further develop our moving-away from 
$h$, this time in terms of units on power spectra plots. explain the unit system
    we are using (ditch $h$ factor because it messes up everything--but
    only briefly summarize the main arguments of Sanchez 2020)}

%! Try to more carefully explain why it doesn't work: h is not a parameter
% over which we emulate--sigma12 is. Therefore we cannot show h in the final
% plots. This is a much subtler issue than the h complaints in the
% previous section.

Conventional emulator calibration entails the historical units of Mpc / $h$,
but if we use instead units of Mpc, then we can distill all of the evolution
parameters into one parameter, $\sigma_{12}$. 


\section{Boltzmann Solvers and CAMB}
\label{sec: boltzmann_intro}

I want to talk about what a Boltzmann solver is and what kinds of equations it is solving.

To hint at what's to come, I start off this section by noting that several cosmological parameters have a fairly unique impact on the shape of the power spectrum, while others have a degenerate impact. Wouldn't it be great if we could know what the power spectrum would look like if we increased parameter $x$? Boltzmann solvers can help us with that.

% Examples of great results from Boltzmann solvers

Three shape parameters of core interest to this paper are the physical density in baryons $\omega_b$ (whose impact on the power spectrum is shown in figure~\ref{fig: omega_b_dependence}), the physical density in cold dark matter $\omega_c$ (figure~\ref{fig: omega_c_dependence}), and the spectral index $n_s$, (figure~\ref{fig: ns_dependence}). The remaining parameters $\sigma_{12}$ and $A_s$, as well as the quantities $z$ and $h$, all shift only the amplitude of the power spectrum. We show the amplitude shift associated with various $\sigma_{12}$ values in figure~\ref{fig: sig12_dependence} and stress that the same plot only needs to be relabeled in order to illustrate the impact of $A_s$, $z$, $h$, etc. 

% What are they solving?

This will mostly just be a theoretical discussion of these solvers. The hands-on stuff comes in the non-introductory section on CAMB.

%%%

In essence, Boltzmann codes solve XXX in order to give us the power spectrum
of any universe characterized by some set of cosmological parameters. For
example, figure~\ref{fig: vary_omega_b} shows the impact of varying the
physical density in baryons, $\omega_b$. 

Several parameters have fairly unique impacts on the power spectrum.
Therefore, we can imagine building a collection of power spectra labeled by
their parameter configurations and comparing our real-world observations to
them. This should allow us to perform parameter inference.

To conclude this section, we make mention of three specific Boltzmann solvers: 
CLASS, CAMB, and CMBFast.
CAMB was written in Fortran and comes with a Python wrapper which we
will use as a starting point (along with GPy) for our Python package
Cassandra-Linear.

CLASS was written in C and also comes with a Python wrapper.
%I will briefly justify our use of CAMB over CLASS.
% Most of our existing emulator pipeline (including the emulator 
% COMET\footnote{\textcolor{green} {Double check this, does COMET actually 
% rely on CAMB?}}) DOES THIS EVEN MATTER?
For this project, we elected to continue with CAMB as the scientists at
OPINAS at the MPE were better equipped to provide support with CAMB.
\textcolor{green}{Furthermore, the CLASS documentation
is not nearly as strong as it is with CAMB, and we already encountered
extreme difficulty simply in recreating results already previously obtained
via CAMB!}

\section{Monte Carlo Markov Chains}

This can be a very brief section, but I want to discuss a little bit of how most modern parameter inference works because it motivates the need for extremely fast power spectrum computation. It provides a sort of conceptual bridge between our ``pure'' goal (quantifying the cosmos) and the nitty-gritty bulk of the paper (optimizing emulator performance).

Metropolis-Hastings algorithm.

We don't know what the true probability distribution of power spectra is. In order to build this distribution with simulation results, we simply draw from the distribution. \textcolor{orange}{Refer to ``Data to Insights'' lecture notes in order to tighten this description.}

\section{Emulation: Basic Principles}
\label{sec: emulation_intro}

To conduct these MCMC analyses, we need several thousands of power spectra. However, if our Boltzmann solvers take on the order of three seconds to run, then these solvers will become the bottleneck of our analysis. \textcolor{orange}{Give some specific numbers for this.}

This motivates the introduction of emulation, basically multi-dimensional interpolation, in order to predict the power spectra. These predictions are orders of magnitude less time-expensive. 

Emulators interpolate across a high-dimensional parameter space. The primary
limitation is that the emulator has to be built with every possible parameter
in mind that an end-user could wish to vary. Yet there is a large number of
different cosmological parameters discussed in the modern literature.
``Currently available emulators only sample a few cosmological parameters,
often with restrictive ranges, and are not applicable to more general
parameter
spaces'' (\cbib{San21}). ``Due to the high computational cost of the required
simulations, [...] current emulators leave out parameters such as the
curvature
of the Universe or dynamic energy models beyond the standard CPL
parametrization'' (\cbib{San21}).

I'll talk a little about different emulators currently available, such as COMET. Some emulate non-linear power spectra, for example, and several even include massive neutrinos. But this thesis will demonstrate that massive neutrinos can be included into our evolution mapping approach, which will be introduced in section~\ref{sec: ev_mapping_intro}.

% (This is good news because the evolution mapping approach greatly simplifies the parameter space, and enhances the accuracy), which is the subject of the next section.

\section{Gaussian Process Regression}
\label{sec: gpr_intro}

% What is a Gaussian Process?

Most emulators are based on a Gaussian Process (GP). A GP is a Gaussian
distribution over functions\footnote
{A GP is the limit of a one-hidden-layer neural network as the number of
neurons approaches infinity.}, which can be interpreted
as the infinite-dimensional generalization of the multivariate normal
distribution. The inference of continuous values with a GP prior
is known as Gaussian process regression, or Kriging. GP regression is a
powerful non-linear multivariate interpolation tool. The computational
complexity of inference and likelihood evaluation within GP regression is cubic
in the number of points. This makes GP regression an excellent companion to
Latin hypercube sampling (LHS), which makes highly efficient use of a limited 
number of samples and whose basic idea will be explained in section~\ref{sec:
lhc_theory}.

Neural networks (NNs) generally need much larger sample sizes to reach
comparable levels of
accuracy. Due to various alterations in the Cassandra-Linear code over its
development, several regenerations of the various emulator data sets were
necessary. This practical constraint motivated the use of a GP for our
emulator. Furthermore, NNs invariably require much more complicated setup and
tuning--for example, in the precise architecture of the network (e.g. nodes
per layer, layer types) as well as the hyperparameters (e.g. learning rate).
By contrast, as we explain in section~\ref{sec: train_emu}, a Gaussian
process regression is highly straightforward to set up and modify. Therefore,
for a demonstration project such as Cassandra-Linear, we elected to base our
emulator on a GP. Please refer to the section~\ref{sec: future_work} for a
continuation of this discussion.

Are there other prediction approaches besides GPs and NNs? IF so, I need to
further justify WHY we’re using GPs.
GPs work best when there are few samples and a lot of parameters, right?
But why is that so? What is the math behind that?


\section{Sampling Approach: Latin Hypercube}
\label{sec: lhc_theory}

I imagine this is going to be an extremely short section. We should motivate why we're using this style of sampling.

What is the theoretical best LHC that we could make?

Besides, can we explain this equation?


\section{Evolution Mapping}
\label{sec: ev_mapping_intro}

\textcolor{blue}{I want to briefly summarize why we can funnel all of the 
evolution parameters through $\sigma_{12}$ in this way. This objective may be
too ambitious for this paper, because I would have to go through each
evolution parameter and derive its evolution nature in a few lines of 
equations.} \textcolor{orange}{Maybe I can just do the same thing as Ariel
recommended with the GPR kernel: ``we played around and found that this
works.''}

(\cbib{San21}) proposes to divide up the full set of cosmological
parameters into two categories: \textit{evolution} parameters $\mathcal{O}_E$
(such as $\omega_b$, $\omega_c$, and $\eta_s$)
affect the amplitude of the power spectrum at a particular redshift, while
\textit{shape} parameters $\mathcal{O}_S$
(such as $\omega_K$, $\omega_\text{DE}$, w(a))
affect the shape of the power
spectrum.

We take, as the evolution mapping relation for the power spectrum, equation 13
from \cbib{San21}:

\begin{equation}
\label{eq: evMapping_pSpectrum}
    \Delta^2_L (k | z, \Theta_s, \Theta_e)
    =
    \Delta_L^2 (k | \Theta_s, \sigma_{12} \left( z, \Theta_s, \Theta_e \right))
\end{equation}\footnote{Varying $z$ has the same effect as varying an
evolution parameter, which is why it appears on the RHS only as an argument to
the $\sigma_{12}$ ``function.'' We write it separately from $\Theta_e$ to
emphasize that $z$ does not describe a property of the Universe, but is
simply used as a proxy here for \textcolor{red}{conformal?} time
\textcolor{red}{elapsed since the Big Bang? (but we can only observe up to
$z = 1100$...)}.}

Why is this scheme important? Evolution mapping greatly simplifies the emulator
implementation. Because we can
funnel all of the evolution parameters through $\sigma_{12}$, we've effectively
collapsed an entire category of parameters to just one parameter. Fewer
parameters means that we get a more accurate emulator.

``At the linear level, all models characterized by identical shape parameters
and the same values of the parameter combinations $b \sigma_{12}(z)$ and
$f \sigma_{12}(z)$ will be identical'' (\cbib{San21}).

Now, for the hiccup, which segues into the next section: this scheme is broken by one parameter, the Universe's
density in neutrinos. (In the next section: why this is so and what we can do
about it.)


\section{Neutrinos and Their Cosmological Impact}

(\cbib{Kiakotou}): ``Neutrinos with masses on the eV scale or below will be a
hot component of the dark matter and will free-stream out of overdensities and
thus wipe out small-scale structures.''

``In general, a larger density of relativistic species leads to a smaller
growth of matter fluctuations'' (\cbib{Zennaro}).

The point of this section is: why is $\omega_\nu$ bad for the
evolution mapping scheme? Because neutrinos exhibit redshift-dependent
damping of the power-spectrum, and therefore affect both the shape and the
amplitude of the power spectrum. Whenever massive neutrinos are present,
the growth factor becomes scale-dependent, which disrupts the
evolution-mapping scheme.

Why do they behave in this way? All neutrinos start off as
relativistic particles in the early Universe, acting as a type of radiation.
But as the Universe continues to expand and cool, the neutrinos behave
increasingly like dark matter.
In this way, the physical density in neutrinos impacts both the shape and the
evolution.

``The popular heuristic formula for the linear theory suppression of the matter
fluctuations by free-streaming $\nu$, $\Delta P(k) / P(k) \approx -8 f_\nu$, is
valid only on very small scales $k > 0.8 h$ / Mpc, However, it is not of
practical use as this is in the strongly nonlinear regime of matter
clustering'' (\cbib{Kiakotou}).

One proposed solution is to treat the neutrinos as a small correction factor
to the results from an anologous cosmology with the same $\omega_m$ but with
$\omega_\nu = 0$. This of course limits the applicability of our emulator to
cosmologies with very small $\omega_\nu$, but this constraint agrees with
current observations (\textcolor{orange}{which?}).

I want to end this section with a vague plan of action: we want to play around with CAMB power spectra to see if there are any simple ways around this limitation in our approach.

%%% New stuff

% what is a MEMNeC?

We already have an approximation for the power spectrum of a massive-neutrino cosmology within the evolution mapping scheme. The $\sigma_{12}$ value that we described earlier is actually the $\sigma_{12}$ value of the model's MEMNeC. can already be approximated within evolution-mapping by slightly altering scheme. \textcolor{red}{Is it fair to say we are adjusting, or was this actually the same scheme as it always was?} We take a MEMNeC and the desired cosmology. The sigma 12 is actually the sigma 12 of the MEMNeC. Then we treat the physical density in neutrinos as a shape parameter along with $A_s$.

  \chapter{CAMB, Initial Setup}

\section{Introduction to CAMB}

CAMB is a Fortran code with a Python wrapper\footnote{
\url{https://github.com/cmbant/CAMB}
} which we will be using for the
entirety of this project.

To introduce the reader to the scope of CAMB, we will now introduce
some basic simulated power spectra along with a summary of the dynamic
parameters which will be of greatest interest to us.

\section{Configuring CAMB for this Project}

\begin{comment}
\textcolor{blue}{
I hope to, in painstaking detail, cover many of the lines of the code that I
have written to interface with CAMB. I will include plots to indicate, at
every step, what incorrect settings cause the power spectrum to look like (or,
for subtler errors, what the error curves looked like compared to Ariel's
results, which I treated as a sort of ``ground truth''). This should also be a
good example to flex my physics interpretation skills: why does this incorrect
setting produce this undesired pattern?}

\textcolor{blue}{You might think that this is sort of an inappropriate 
section
for a master's thesis (especially since I have in mind that this be a lengthy 
section), but I would like to include it unless you feel very strongly. After 
all, I spent several months of the project debugging at least ten different 
ways that slight and major errors in the various settings led to 
irreconcilable results.}
\end{comment}

\textcolor{orange}{Ariel recommends
just talking about the correct lines, don't talk about what happens when
they're wrong.}

%For example, one parameter that tripped me up for a while:
The impact of some parameters can be quite subtle, especially (we imagine)
for users unfamiliar with the nuances of neutrino physics.

Consider the neutrino mass 
hierarchy: the options are degenerate, normal, and inverted. The CAMB 
documentation annotates this parameter as ``(1 or 2 eigenstate 
approximation),'' but this is somewhat unclear. Is the degenerate hierarchy 
the single mass eigenstate approximation? Do both normal and inverted 
hierarchies involve two eigenstates? Besides, even if this description were
literally accurate, why does use of the normal mass hierarchy lead to
incorrect results?

% C'mon, Ariel's GOT to let me have this plot. It's a big open question!

%In figure \ref{fig: spectrum_type}, we can see that requesting of the wrong
%power spectrum type can in some low-$\omega_\nu$ cases yields errors so low
%that we might accidentally overlook them. This error pattern is easily
%recognizable and is a consequence of the definition of the power spectrum: the
%Fourier transform  of the two-point correlation function. ...Okay, I'm still thinking about this. I don't understand %yet, but I'll be sure to ask you if I'm still struggling about it.

Another paragraph I want to have in this section: stress the part of the evolution mapping introduction, that the $\sigma_{12}$ value we're using to describe the model is actually the $\sigma_{12}$ value of the model's MEMNeC! This is so important and confusing that maybe I'll even recapitulate again later in the section on the generate\_emu\_data.script.

%s Now discuss individual settings

In principle, the accuracy boost should increase the variance in our models but should not affect the bias. The line is

\verb|pars.Accuracy.AccuracyBoost = 3|

\textcolor{orange}{But what does it do?}

\verb|pars.NonLinear = camb.model.NonLinear_none|

% Why is CAMB set up this way anyway? Why do I have to put in a model object 
% instead of simply flipping a Boolean? {probably not a relevant question for
% this thesis.}

\verb|pars.WantCls = False|

``\verb|WantCls| – (boolean) Calculate C\_L.'' \textcolor{orange}{This entry 
of
the CAMB documentation is NOT helpful. Looks like we’ll have to do some
detective work.}

\verb|pars.WantScalars = False|

``\verb|WantScalars| – (boolean) Calculates scalar modes.'' We should briefly 
talk about what they are, then say ``we are not interested in these, let's 
save time by skipping the computation.''

\verb|pars.Want_CMB = False|

``Calculate the temperature and polarization power spectra.'' We're not 
interested in these quantities; in order to apply computational resources more 
efficiently, we should avoid these calculations and therefore set this to 
\verb|False|.

\verb|pars.DoLensing = False|

``Include CMB lensing'' but \textcolor{red}{why don't we care about this? Does
this not affect the power spectrum results in any way that we can appreciate
within the confines of the MCMC experiment that we're testing here?}

\verb|pars.YHe = 0.24|

\verb|pars.Accuracy.lAccuracyBoost = 3|

\verb|pars.Accuracy.AccuratePolarization = False|

%s Small things

Beware the neutrino settings. The effective number of massive neutrinos is about 3.027

\section{Convenience Functions}

% camb\_interface.py

\textcolor{blue}{The purpose of this section is to anticipate the Python
package Cassandra-Linear. The settings we describe in the previous section
still technically correspond to lines of code of course, but here we tie these
individually-treated lines into bigger functions that users can quickly use to
get results from CAMB!}

\section{Verifying Our Settings}

%! This is weird, and perhaps inappropriate for a thesis document.

To test this setup, we compare our results with those of Ariel S\'{a}nchez 
(within the native Fortran framework of CAMB) and Andrea Pezzotta for the
first seven Aletheia cosmologies and four $\omega_\nu$ physical densities in
neutrinos (\textcolor{green}{approximately} 0, 0.0006, 0.002, 0.006.), for a 
total of 28
models. The errors are miniscule and recorded in figure XXY. After verifying
the accuracy of our code in this way, we proceed to experiment with the power
spectra in order to explore solutions to the evolution mapping problem.

% Aletheia Model 8 has not been integrated into our code suite yet. We can’t handle its DE weirdness. Instead of bringing up and then discarding model 8, we shouldn’t mention it at all.


  \chapter{Expansion of the Parameter Space}
\label{chap: A_s}

Now that we have a comprehensive set of convenience functions with which to
configure CAMB, we can easily obtain the theoretical power spectra we need in
order to demonstrate our solution to the neutrino difficulties introduced in
section~\ref{sec: neutrino_problem}. This chapter will motivate the inclusion 
of the additional parameter $A_s$ in the evolution mapping framework.
Because the focus of this thesis is regression rather than theory,
we will merely show that $A_s$ contains information relevant to the impact of
massive neutrinos. Then, the GPR should, in principle, be able to optimize its
use to predict power spectra.

% Maybe it would make more sense to have the large cass-L chapter focus on the
% creation of a massless-neutrino emulator and THEN a smaller chapter focusing
% on all the changes necessary for it to become a massive-neutrino emulator.
% BUT! As of 25-08-23, I'm running way behind on writing actual content for
% this thesis. I can't risk redoing all the section headers again. We're
% going to proceed under the current scheme and MAYBE allow ourselves to redo
% it shortly before submission.

\section{Equation to be Solved}

% Redo section title

\textcolor{orange}{Remember to explain \textit{why} the prediction of these 
asymptotes means that $A_s$ will capture most of the unruly behavior of 
neutrinos. What theory motivates such a judgment? Or are we only guessing?}

We begin with the second proposal from section~\ref{sec: neutrino_problem}:
to approximate the power spectrum of a massive-neutrino cosmology, we
combine the
power spectrum of its MEMNeC with an approximation of the ratio
between the two:

\begin{equation}
\label{eq: MEMNeC_approx}
P_\nu(k) \approx \de^* (k) P_0 (k)
,\end{equation}

The essence of this chapter is to increase the accuracy in our $P_\nu(k)$
predictions by estimating the true $\de(k)$. Specifically, we are interested
in how the true $\de(k)$ deviates from $\de^*(k)$. Therefore, we pay special
attention to the quantity:

\begin{equation}
\label{eq: ee}
\ee (k) \equiv \frac{\de(k)}{\de^*(k)}
\end{equation}

To simplify the discussion, we will concentrate on the small-scale limit of 
this ratio $\el$

\begin{equation}
\el \equiv \lim_{k \rightarrow \infty} \ee(k)
\end{equation}

If we can estimate $\el$ accurately, then in principle our task 
is complete, because the slope toward smaller $k$ values is
even and analytically predictable. \textcolor{green}{Can I refer the reader to
 a paper on this?}

\section{Proposed Fitting Functions}
 
In order to explore this limit concretely, we will use the Aletheia models as
our test cases. This means that we can rewrite definition~\ref{eq: ee} as:

\begin{equation}
\label{eq: ee}
\ee_i (k) \equiv \frac{\de_\text{model i}(k)}{\de_\text{model 0}(k)}
\end{equation}

\textcolor{orange}{This section, like the ``Convenience Functions'' section
from chapter 2, should anticipate CL: we should talk about the
code that we have written in order to explore these ratios, which one can find
in} \verb|camb_interface.py|.

To study the behavior of $\ee_i$, we developed the \verb|camb_interface.py|
function \verb|model_ratios|. This function accepts a single snap
index\footnote{Remember that there are four snaps for each model, and snap 
four always corresponds to $z = 0$} and a set of power spectra nearly in the
format returned by \verb|boltzmann_battery|, but without the $\omega_\nu$ 
layer.\footnote{For explanations of the
remaining function parameters, we refer the reader to the docstring} Tt
computes and plots all of the $\ee_i$ spectra in the input set. We show an
example output in figure~\ref{fig: model_ratios_demo}. This function
is recommended for any users seeking to improve the characterization of
$\ee_i$.

% The following plots were generated with divergence_asymptotes.ipynb
\begin{figure}[ht!]
    \begin{subfigure}{0.45 \textwidth}
    \centering
 		\includegraphics[width=\textwidth]{chap3/model_ratios}
 		\cprotect\caption{Example output of \verb|model_ratios|.}
 		\label{fig: model_ratios_demo}
    \end{subfigure}
    \begin{subfigure}{0.45 \textwidth}
    \centering
 		\includegraphics[width=\textwidth]{chap3/predictor_success}
 		\caption{With asymptote predictions.}
 		\label{fig: ee_prediction_demo}
    \end{subfigure}
        \centering
    \caption[$\ee$]
    		{Model ratios
    		\textcolor{red}{Would it have been better if I just included the
    		right plot and no left plot?}}
    \label{fig: model_ratios}
\end{figure}

\textcolor{orange}{The above paragraph only describes the function in the
case ``massive=`x'''}

\textcolor{green}{Both plot titles need to be redone.}

After experimenting with different functions and parameters, we find the
following fitting function:

\begin{equation}
\hat{\ee}_i = C \, \omega_\nu \, \ln \left( \frac{A_{s, i}}{A_{s, 0}} \right)
.\end{equation}

We demonstrate its predictions in figure~\ref{fig: ee_prediction_demo}.

We also show the errors for the Aletheia models in
figure~\ref{fit_errors_Aletheia}.

\section{Testing Functions}

To verify that our approach has general applicability, we now test the
asymptotic fit on a much broader selection of models.

\textcolor{orange}{What if we tested this fitting function on an independent
set of models with a consistent but different set of shape parameters?}

First, we created the function \verb|get_random_cosmology| function, which
accepts an $\omega_\nu$ value and returns a cosmology with randomized
evolution parameters over the following ranges:~\ref{tab: fit_test_params}.
All of the shape parameters are still consistent with Aletheia model 0.

\begin{table}[ht!]
\centering
\begin{tabular}{l|l|l}
\hline
Parameter & Minimum Value & Maximum Value \\ \hline
$\omega_\text{DE}$\footnotemark & 0.1 & 0.5 \\
$\omega_K$ & -0.05 & 0.05 \\
$w_0$ & -2.0 & -0.5 \\
$w_a$ & -0.5 & 0.5 \\
$A_s$\footnotemark & $5.003 \cdot 10^{-10}$ & $1.484 \cdot 10^{-8}$  \\
\end{tabular}
 \cprotect\caption[Parameter Ranges for Random Test
 	Cosmologies]{Parameters which we vary for the purpose of testing the
 	generality of fit~\ref{eq: fit}, and their domains.
 	\textcolor{red}{Is it poor form to have a footnote that looks like an
 	exponent? How should I do it differently?}}
 \label{tab: fit_test_params}
\end{table}

\addtocounter{footnote}{-1}

\footnotetext{Unfortunately, $\omega_\text{DE}$ is not accepted as an
input by CAMB's \verb|set_cosmology|. To vary $\omega_\text{DE}$ in this way,
we fix $\omega_b$ and $\omega_c$ and select a value for $\omega_K$. Then, we
vary $h$--according to equation~\ref{eq: h_to_omega}, this should only
affect $\omega_\text{DE}$.}

\footnotetext{These values were rounded to three significant figures. Refer
to the code for greater precision.}

These cosmologies cover a wide range of $\el$ values, as shown in
figure~\ref{fig: random_battery}. The errors for the fits are shown in
figure~\ref{fig: random_battery_errs}.

To further test our solution, we also created the function
\verb|get_As_matched_cosmology|, which randomizes the cosmologies in the same
way except for fixing $A_s$ at the value for model 0.
figure~\ref{fig: degenerate_battery}. The errors for the fits are shown in
figure~\ref{fig: degenerate_battery_errs}.

This third test demonstrates that the imperfections in our fit cannot be
attributed solely to the form of equation~\ref{eq: fit};  
since all of the cosmologies share $\omega_\nu$ and $A_s$ values, and since
the asymptotes were nevertheless not all identical, the
asymptote must have some dependence also on one of the other parameters that
we varied.  

However, the larger errors that we see in
figure~\ref{fig: random_battery_errs} suggests that our fit may only
approximately describe the relationship
between $A_s$, $\omega_\nu$, and the small-scale suppression of the power
spectrum. A symbolic regression investigation could prove highly effective at
resolving this ambiguity by efficiently searching out improved formulas. 
However, we do not consider this a promising avenue
for the continuation of this work (see section~\ref{sec: future_work} for our
recommendations); the error
is so low here that we do not believe the imperfect predictions to
significantly detract from the performance of the emulator.

\section{Summary of Findings}

% I'm not sure if this section will survive, we'll see how big the conclusion
% here ends up being.

In this chapter, we have illustrated how the scalar mode amplitude $A_s$
contains information of great relevance to the application of evolution
mapping to the emulation of massive-neutrino power spectra. As a result of 
these demonstrations, we know to construct 
our massive-neutrino emulator emulator over the six cosmological
parameters $\omega_b$, $\omega_c$, $n_s$, $\sigma_{12}$, $A_s$, and
$\omega_\nu$. In other words, to predict a power
spectrum, our emulator will accept as an input a vector describing
these six parameters. Conceptually, adding $A_s$ to the set of
emulation parameters allows us to treat $\omega_\nu$ like a shape
parameter.

Expansion of the parameter space represents the single most 
important novel step in this work. The remaining chapters will
focus on the integration of this modified evolution mapping framework
into a beginner-friendly emulation code.

\textcolor{blue}{Plot to do: add errors for original Aletheia set; add two
plots for random cosmology experiment; add two plots for fixed
A\_s experiment.}
  \chapter{Cassandra-Linear: a Python Package}

\textcolor{blue}{This will be a rather long, dry, and technical section with 
subsections based on each of the core scripts making up the Python package \
that I have been developing. It will in some ways paraphrase and summarize the 
documentation, explaining the basic use of the package as well as important 
limitations. Of course, fairly early on in this section, there will be a 
footnote linking to my GitHub repository, which will be made public once I'm 
ready to hand in this thesis.}

\textcolor{blue}{Even if my code doesn't end up in a bigger repository, I 
nevertheless think that it's important to the scientific process that I 
describe in detail the code that I have written. Besides, if any readers want 
to experiment specifically with the ideas discussed in this thesis, they may 
find my code more accessible because it is, in a sense, "single-purpose"--that 
is to say, written almost exclusively to investigate the topics of this 
paper.}

\textcolor{orange}{It only takes a couple of sentences, but we also need to 
describe how to install the package.}

\section{Conceptual Outline}
\label{sec: cassL_concept}

\textcolor{blue}{The point of this section is to introduce the reader to the
basic structure of the emulator pipeline. Most of the discussion will follow
from a flow chart showing the various conceptual pieces of the emulator. Also
in this section, I will introduce various choices made (e.g. 5000 training
spectra, 300 points for each spectrum) WITHOUT justifying any of them--later,
in chapter~\ref{chap: disc_and_conc}, we'll justify these decisions and
talk about the consequences of alternate settings.}

\textcolor{orange}{This section is getting kind of ambitious. I may break it
out into a separate chapter, especially because the nature of the discussion
is fairly different from the rest of this chapter (here very high-level
and conceptual, later technical and programming-oriented).}

% We're gonna need a flow chart for this puppy. That should be priority number
% one, because the writing should constantly refer back to it.

\subsection{Design Choices}

By default, each Latin hypercube consists of $N_s = 5000$ entries. For 
justification of this quantity, please refer to
section~\ref{sec: num_samples}.

$N_k = 300$.

\subsection{Creation of Separate Emulators}
\label{sec: 2emu_intro}

This is a brief introductory section motivating the creation of two emulators:
unlike the other five parameters, the physical density in massive neutrinos 
has a hard lower bound. This creates some minor regression problems (appeal to 
boundary effects / danger of extrapolation).

To simplify the user experience, this two-emulator solution lives ``under the
hood'' and by default \textcolor{orange}{will be} hidden behind an interface
which automatically queries the correct emulator given some user-input
cosmology.

To justify our decision and to quantify the improvement from this approach, we
have prepared \textcolor{orange}{some} plots in section~\ref{2emu_improvement}.

Conclude this section by saying that this integration of multiple emulators into one user-facing script can be further exploited with, for example, different emulators for different neutrino mass hierarchies.

\subsection{Unit LHCs}
\label{sec: unit_lhc}

The \verb|pyDOE2| function that we will use to generate our LHCs
returns a result with entries that range from zero to unity. These unit LHCs 
will eventually be used to create a set of cosmological
configurations which will act as the $X$ data set when we train our emulator.

In order to these LHC entries into cosmological parameters that we can input 
into CAMB, we will need to rescale them with our priors for each parameter.
However, instead of rescaling immediately, we will rescale later: 
\textcolor{orange}{The benefits of rescaling later are. This is useful not
just for changing priors but even for changing the way that we sample a
particular prior (for example, $\sigma_{12}^2$ sampling--although we did not
get it to work, the infrastructure is still there for whoever comes along in
the future.}
 
We describe the building of the unit LHC in section~\ref{sec: build_lhc}
and its rescaling (as well as the construction of the $Y$ data set) 
in~\ref{sec: generate_emu_data}.


\subsection{Default Priors}

% One reason this goes here: priors show up in almost every script,
% so there isn't a great cass-L subsection in which to put this.

The priors that we use correspond to those currently used by COMET. \textcolor{orange}{Should I spend any time defending this choice here, or should I put all of the defense in section~\ref{sec: priors}?}

\textcolor{orange}{Table goes here.}

\subsection{Emulation over Uncertainties}

For a further step of accuracy, we can add a third data set to our pipeline
and introduce a second layer of emulation.

Up to this point, our pipeline has included a training set and a testing set.
If we add a validation set, then we can train a second emulator over the
errors associated with the first emulator's performance on this validation
set.

Within the current (as of \textcolor{orange}{24.08.2023}) setup of
Cassandra-Linear, we typically generate two Latin hypercubes for each
emulator. The first represents our training set and an emulator cannot be
produced without it. The second one represents our validation set.

\textcolor{orange}{In a future
release of Cassandra-Linear, this set will be used to train an ``uncertainty''
emulator, which will train over the errors from the main emulator in order to
provide the user with an uncertainty estimate for any cosmology located within 
the space of priors over which the main emulator was trained.} 

\textcolor{orange}{However, this functionality has not yet been implemented. 
Currently, we are 
using the validation hypercube more as a test hypercube: we compute the uncertainties at discrete points and examine these uncertainties (e.g. with
scatterplots and histograms) to assess the performance of the emulator.}

\section{Building the Latin Hypercube}
\label{sec: build_lhc}

% lhc.py

\textcolor{blue}{This section will briefly recapitulate some of the Latin hypercube sampling
explanation from the introduction. Then we will talk about how this sample is
procured in practice, and what our ``solution'' (inelegant though it may be)
is to obtaining a sample with a decent minimum separation between the points
(remember to briefly repeat why this is important).}

In this section, we will discuss the building of LHSs, which within the
Cassandra-linear is handled by the \verb|lhs.py| script.
By ``LHSs,'' we are here referring exclusively to the
sample of cosmologies for which we want to compute power spectra.
However, as mentioned in section~\ref{unit_lhc}, at this stage we are not
yet committing to a particular set of priors. Instead, each axis is sampled
from the uniform distribution $\mathcal{U} (0, 1)$.

We'll also describe the switch from a rescaled Latin hypercube to a unit Latin hypercube that we later interpret differently according to the specified priors.

Each entry in our LHS will be either a four- or six-dimensional vector, 
depending on the emulator's support for massive neutrinos. When building a
massless-neutrino emulator, a single LHS entry will describe the $\omega_b$,
$\omega_\text{CDM}$, $n_s$, and $\sigma_{12}$ values for that cosmology in 
this order. When building a massive-neutrino emulator, a single LHS entry will 
additionally describe the $A_s$ and $\omega_\nu$ values for that cosmology, 
again in this order.

To get started generating LHSs, we integrated demonstration code written by 
Daniel Farrow \text{red}{(and who else?) add these people to the Danksagung} 
during the CAKE 2021 workshop. We begin by invoking the \verb|lhs| function of
\verb|pyDOE2| a number of times. Each time we get a random new unit LHC, whose
spacing we analyze with the \verb|cdist| function from
\verb|scipy.spatial.distance|.  Recall from section~\ref{sec: lhc_theory} that
our sample more evenly covers the sample space if the minimum separation 
between the sample points increases. \textcolor{green}{Uneven separations will
result in excess error depending on the location of the cosmology in parameter
space}. Therefore, from among the numerous calls of \verb|lhs| we select the
hypercube sample with the largest minimum separation.

In order to maximize the number of random hypercubes generated per unit of
wall time, we have written a multithreaded function called
\verb|multithread_unit_LHC_builder| \textcolor{orange}{Keep your eye on this
one, I'm thinking about condensing lhc.py to just a couple of functions.}
Through the function parameter \verb|num_workers| the user is free to assign
any number of CPU threads to the task, so in principle the computer can be
wholly dedicated to the optimization of the LHS. This function represents an
unending query for LHSs. The user will be notified via command
line printout whenever an LHS has been generated whose minimum separation is
higher than the previous record value. Whenever such a superior LHS is
encountered, the function writes the LHS to a file and continues. Therefore,
the user may run this script in the background and terminate whenever. The
function always writes LHSs to the same file, so each new record-setting LHS
overwrites the old one. When the user terminates the function, whatever output
file remains represents the best LHS seen since the function was first called.

\begin{comment}
The \verb|cdist| function can be re-used to compare LHSs loaded from 
different
files. However, since there is generally little reason to keep old LHSs
(except, perhaps, to reconstruct specific emulators), it reduces clutter to
simply continue overwriting the same file. Therefore, the function
\verb|multithread_unit_LHC_builder| also includes a parameter
\verb|previous_record|, which is recommended whenever the user would like to
stop the function and then resume it later. In such a case, the parameter
should be set to the \verb|cdist| value of the exis}
\end{comment}

% I've got a better idea: the function should automatically ask the user if
% he's sure, in the case that we find a file under the name under which we
% intend to write. If the user is sure, we take the cdist of that existing
% file and use it as the previous record!

% we multithreaded the Python script to spawn 12 workers on an 11th gen Intel i7-11700 @ 2.50 GHz. IF YOU ARE USING THE WORK DESKTOP

As of the current (as of \textcolor{orange}{24.08.2023}) version of the
Cassandra-Linear code, this brute-force approach is the method used to build
the Latin hypercube samples eventually used to train the emulators. To 
understand the computational and wall-time costs associated with our
brute-force solution, we include plots XXY and XXY.

\begin{comment} % The following paragraph says this stuff better
we left the system to run for three consecutive days. In this time, the 
largest minimum separation that we generated was approximately 0.08022.  
Recall from section sec_B1 that the theoretical best possible value for this 
setup is approximately 0.24183. It would have been more meaningful if you had 
counted the total number of function calls, but it isn’t too late to set up 
such a run. So, even after assigning a relatively large amount of compute to 
this brute force solution, we fail to obtain an LHC of even a third of the 
best minimum separation.
\end{comment} 

Remember from section~\ref{sec: lhc_theory} that we have an equation for the
best possible minimum separation that we can achieve with a Latin hypercube
of some given dimension and total number of samples. Let us consider a massless-neutrino emulator (a total of four dimensions per cosmology vector)
trained over five-thousand CAMB spectra. If we plug in $d = 4$ and $N=5000$,
we get XXY. Similarly, for our massive-neutrino emulator (a total of six 
dimensions per cosmology) trained over the same number of spectra, we find a 
theoretical best minimum-separation of XXY. 

It is based on figures XXY and XXY that we may claim that our approach is
inefficient and not well-suited to the approach of finding LHCs with large
minimum separations. As mentioned in section~\ref{sec: lhc_theory},
we value a high minimum separation because it means we are sampling the
space of cosmologies evenly rather than oversampling particular regions.
Therefore, we expect the consequence of a low minimum separation in the LHC to
be a higher error variance as well as a slightly higher average bias.

% One could argue that the variance in oversampled regions will decrease to
% compensate, right? I need to clarify that the decrease in variance in the
% oversampled regions would not be able to compensate the increased variance
% in the undersampled regions, because the space of power spectra is
% continuous, so closely spaced points reveal less about the true function
% than well spaced points.

Later, we will test our expectations by comparing emulators with different
minimum separations in their LHSs in section~\ref{sec: error_from_lhc}, to
which we defer remaining concerns about our approach to LHC generation.
Additionally, we will explore the question of superior methods in
section~\ref{sec: future_work}.

\section{Integrating Evolution Mapping}
\label{sec: generate_emu_data}

% generate\_emu\_data.py

\textcolor{blue}{This section will describe the process going from a Latin 
hypercube to a set of CAMB power spectra. We will describe how an array of six 
parameter values is fleshed out into a params object understood by CAMB. We 
will describe the procedure of modifying $h$ and $z$ until we arrive roughly 
at the $\sigma_{12}$ value that we desire, as well as the process of writing 
the \textit{actual} $\sigma_{12}$ value that we obtained back to the original 
hypercube. Why is this important? Because the Latin hypercube is used again 
later in the user\_interface.py script for the purpose of training the 
Gaussian emulator object.}

Ariel’s idea for speeding up the program: use a best-guess z which only approximates the desired sigma12. Then, when we calculate the power spectrum using this redshift, get the actual sigma12 value from CAMB and replace the LHC entry with the true value.
We’ve implemented this now. We should definitely talk about how little of a difference that this makes. We should make error plots (comparing the two approaches with hyper cube as a control), but I strongly suspect the error will be negligible.


\section{Training the Emulator}
\label{sec: train_emu}

% train\_emu.py

This section will focus on the particular lines of GPy that we used, as well
as the various data-cleaning and normalization statements that we used.
Normalization will be a really important topic. I want to explain some of
the theory behind why the emulator performs poorly with values outside of this
[0, 1] range.

I don't know how to justify this kernel that I got from Alex via AndreaP...

\section{Accessing and Using the Emulator}

% user\_interface.py

The content of this section is still relatively uncertain. Since I am still tweaking some of the emulator's settings, I haven't spent too much time on a script dedicated purely to simplifying the interaction between the user and the emulator \textit{object} itself.

Anyway, the hope is to provide some clear and simple descriptions of what functions the user should turn to in order to get started predicting power spectra using the results of this thesis. The functions can either create a new emulator object and train it or load an existing one with easy-to-understand functions for interacting with the object.

Here, we will talk about the details of the training as well as the importance of normalization. Since this is still technical, it may not be appropriate for a UI script. So, I am considering breaking out a new script (and therefore a new section), maybe called ``train\_emu.py.''

  \chapter{Implementation Details}
\label{chap: implementation}

\textcolor{blue}{This will be a rather long, dry, and technical section with 
subsections based on each of the core scripts making up the Python package \
that I have been developing. It will in some ways paraphrase and summarize the 
documentation, explaining the basic use of the package as well as important 
limitations. Of course, fairly early on in this section, there will be a 
footnote linking to my GitHub repository, which will be made public once I'm 
ready to hand in this thesis.}

\textcolor{blue}{Even if my code doesn't end up in a bigger repository, I 
nevertheless think that it's important to the scientific process that I 
describe in detail the code that I have written. Besides, if any readers want 
to experiment specifically with the ideas discussed in this thesis, they may 
find my code more accessible because it is, in a sense, ``single-purpose''--that 
is to say, written almost exclusively to investigate the topics of this 
paper.}

\textcolor{orange}{It only takes a couple of sentences, but we also need to 
describe how to install the package.}

\section{Building the Latin Hypercube}
\label{sec: build_lhc}

% lhc.py

In this section, we will discuss the building of LHSs, which within CL is 
handled by \texttt|lhs|. As mentioned in section~\ref{sec: lhc_outline},
we first create a unit LHS and then later rescale it according to a set of
priors.

Each entry $\bm{x}$ in our LHS $\matr{X}$ will be either a four- or
six-dimensional vector, 
depending on the emulator's support for massive neutrinos. When building a
massless-neutrino emulator, each $\bm{x}$ will describe the $\omega_b$,
$\omega_c$, $n_s$, and
$\tilde{\sigma}_{12}$ values for a different cosmology in 
this order. When building a massive-neutrino emulator, each $\bm{x}$ will 
additionally specify the $A_s$ and $\omega_\nu$ values for that cosmology, 
again in this order.

To get started generating LHSs, we integrated demonstration code written by 
Daniel Farrow \textcolor{orange}{(and who else?) add these people to the 
Danksagung}. We begin by invoking the \texttt{lhs} function of
\texttt{pyDOE2} a number of times. Each time we get a random new unit LHC, 
whose spacing we analyze with the \texttt{cdist} function from
\texttt{scipy.spatial.distance}.
Recall from section~\ref{sec: lhc_theory} that
our samples more evenly cover the sample space if the minimum separation
$s^*$ increases. Therefore, from among the numerous calls of \texttt{lhs} we 
select the LHS with the largest minimum separation.

In order to maximize the number of random hypercubes generated per unit of
wall time, we have written a multithreaded function called
\verb|multithread_unit_LHC_builder| \textcolor{orange}{Keep your eye on this
one, I'm thinking about condensing lhc.py to just a couple of functions.}
Through the function parameter \verb|num_workers| the user is free to assign
any number of CPU threads to the task; in principle, a CPU can be
wholly dedicated to the optimization of the LHS.
\verb|multithread_unit_LHC_builder| does not contain a return statement but
instead triggers an unending query for LHSs. The user is notified via command
line printout whenever an LHS has been generated whose $s^*$ exceeds
the previous record. Whenever such a superior LHS is
encountered, the function writes the LHS to a file and continues. Therefore,
the user may run this script in the background and terminate whenever. The
function always writes LHSs to the same file, so each new record-setting LHS
overwrites the old one. When the user terminates the function, whatever output
file remains represents the best LHS seen since the function was first called.

%%% It's just too much detail for a master's thesis, no one will care
\begin{comment}
The \verb|cdist| function can be re-used to compare LHSs loaded from 
different
files. However, since there is generally little reason to keep old LHSs
(except, perhaps, to reconstruct specific emulators), it reduces clutter to
simply continue overwriting the same file. Therefore, the function
\verb|multithread_unit_LHC_builder| also includes a parameter
\verb|previous_record|, which is recommended whenever the user would like to
stop the function and then resume it later. In such a case, the parameter
should be set to the \verb|cdist| value of the exis}
\end{comment}
%%%

% I've got a better idea: the function should automatically ask the user if
% he's sure, in the case that we find a file under the name under which we
% intend to write. If the user is sure, we take the cdist of that existing
% file and use it as the previous record!

% we multithreaded the Python script to spawn 12 workers on an 11th gen Intel 
% i7-11700 @ 2.50 GHz. IF YOU ARE USING THE WORK DESKTOP

As of 2 October 2023, CL relies on this brute-force approach to build the
LHSs that are eventually used to train the emulators. To understand the 
computational and wall-time costs associated with this solution, we include 
plots~\ref{fig: function_calls} and~\ref{fig: wall_time}.

\begin{comment} % The following paragraph says this stuff better
we left the system to run for three consecutive days. In this time, the 
largest minimum separation that we generated was approximately 0.08022.  
Recall from section sec_B1 that the theoretical best possible value for this 
setup is approximately 0.24183. It would have been more meaningful if you had 
counted the total number of function calls, but it isn’t too late to set up 
such a run. So, even after assigning a relatively large amount of compute to 
this brute force solution, we fail to obtain an LHC of even a third of the 
best minimum separation.
\end{comment} 

Remember from section~\ref{sec: lhc_theory} equation~\ref{eq: best_lhs_sep},
which tells us the best possible $s^*$
given dimension $d$ and total number of samples $N_s$. Let us consider a 
massless-neutrino emulator (a total of four dimensions per cosmology vector)
trained over five-thousand CAMB spectra. If we plug in $d = 4$ and $N=5000$,
we find $s^*_\text{best} \approx 0.1189$. Clearly, the brute force method
achieves only a comparatively low $s^*$, even over the span of multiple days.

Based on figures~\ref{fig: function_calls} and~\ref{fig: wall_time}, we
claim that our approach is inefficient and not well-suited to the approach of 
maximizing $s^*$. Since suboptimal $s^*$ values imply unevenness in the
coverage of the space of cosmologies, we expect low $s^*$ values to impact
the emulator by increasing variance in the errors, because the worst errors
will be significantly worse, while the best errors will be slightly better in
oversampled regions. We also expect the average error to increase slightly,
as the oversampled regions should benefit less than the undersampled regions
suffer.

% One could argue that the variance in oversampled regions will decrease to
% compensate, right? I need to clarify that the decrease in variance in the
% oversampled regions would not be able to compensate the increased variance
% in the undersampled regions, because the space of power spectra is
% continuous, so closely spaced points reveal less about the true function
% than well spaced points.

We will test these expectations in section~\ref{sec: error_from_lhc} by
varying the $s^*$ of the LHS with which we train the emulator.
\textcolor{orange}{We will explore the question of superior methods in
section~\ref{sec: future_work}}.

\section{Rescaling the LHS}
\label{sec: lhc_rescale}

%s Introduce the build_cosmology script

Now that we have a unit LHS $\matr{X}_u$, we need to scale it 
so that each axis, which currently runs from zero to one, runs along a range 
dictated by one of the priors. This scaling is handled by the \texttt{ged}
function \verb|build_cosmology|. It accepts as parameters the values of
$\omega_b$, $\omega_c$, $n_s$, $\sigma_{12}$, $A_s$, $\omega_\nu$, and, 
optionally, a dictionary of priors. If this
dictionary is not provided, the cosmological parameters are assumed to
already have been scaled. When building a massless-neutrino emulator,
the information $\omega_\nu = 0$ and $A_s = A_s(\text{Aletheia model 0})$ is
automatically provided to \verb|build_cosmology| by
\verb|fill_hypercube|. 
% We really should redo build_cosmology so that it assumes indices match to
% the same parameters every time! This is how the rest of the code works,
% after all.

The precise form of the scaling is given by the formula for transforming a
random variable $x \sim \mathcal{U}(0, 1)$ to a random variable
$x' \sim \mathcal{U}(a, b)$:

\begin{equation}
\label{eq: scaling}
x' = x (b - a) + a
\end{equation}

After scaling the parameters, \verb|build_cosmology| finishes bridging the gap
between the LHS and the CAMB \verb|pars| object (as introduced in
chapter~\ref{chap: CAMB_setup}) by using default values to fill in the
remaining values demanded by CAMB. For example, $H_0$, $w_a$ and $w_0$ are
not parameters over
which we train the emulator, but CAMB requires that they be specified before
a power spectrum can be calculated. In all such cases, we use Aletheia model
0 as default values. \textcolor{orange}{I'm not referring to table 1.2 here
because I would have to expand it with parameters not essential to this work,
so I would have to redo the captions for fig 1.1 and table 1.2...}

So long as the power spectrum's value for $\sigma_{12}$ agrees with the
prior-scaled value from the LHS, it should not matter that 
we use the model 0 values, as these are evolution parameters.
\textcolor{red}{Or should it? If this logic really held, shouldn't we be able 
to modify $w_a$ and $w_0$ in order to get the correct value of
$\tilde{\sigma}_{12}$? But these are evolution parameters, so why don't we?}

\section{Integrating Evolution Mapping}
\label{sec: generate_emu_data}

% generate\_emu\_data.py

%s Now talk about fill_hypercube, a central function of this script

\verb|fill_hypercube| is the central function of \texttt{ged}. It iterates 
through the rows $\bm{x}$ of the unit LHS
$\matr{X}_u$, packages them into a fully-specified cosmology using
\verb|build_cosmology|, and passes the cosmology to one of the 
evaluation functions. \texttt{ged} provides two evaluation functions, which 
combine the CAMB calls from \texttt{ci} with the
principles of evolution mapping: \verb|direct_eval_cell| relies on
\verb|evaluate_cosmology| while \verb|interpolate_cell| relies on
\verb|cosmology_to_Pk_interpolator|. These two options are designed to give
the same results and will differ at a level insignificant to the conclusions
of this work. The emulator pipeline uses the direct evaluation approach by
default.

%s What exactly does it mean here to integrate evolution mapping into the 
%s code?

Recall from chapter~\ref{chap: CAMB_setup} that CAMB does not accept
$\sigma_{12}$ as an input. The importance of \verb|direct_eval_cell| and 
\verb|interpolate_cell| is in circumventing this problem.
After \verb|build_cosmology| returns a complete cosmology dictionary, these
functions computer the MEMNeC. Then, they request CAMB power spectra for the
MEMNeCs at 150\footnote{150 is the maximum number of redshifts that CAMB will
accept in a single call.} linearly-spaced redshifts in the
interval [0, 10]. \textcolor{orange}{log spacing would have been better for
redshift}. This interval suffices to capture the reddest galaxies
in our galaxy redshift surveys. \textcolor{green}{citation}. Next, these
functions request a one-dimensional interpolator from \texttt{scipy} with
the $\tilde{\sigma}_{12}$ values as the $x$ and the $z$ values as the $y$.
By passing the desired $\tilde{\sigma}_{12}$ value to the interpolator, we
can in principle find the redshift at which we need to call CAMB.

%s Bonus section that doesn't really fit anywhere specific: speed-up

Since we are estimating this redshift from an interpolation over 150 power
spectrum samples, the $z_\text{interp}$ returned by our interpolator does not
\textit{exactly} match the theoretical $z_\text{exact}$ at which the
power spectrum would exactly match the desired input value,
$\tilde{\sigma}_{12}(z_\text{exact})$. In order to speed up \texttt{ged},
which is by far the most time-intensive step in the emulator pipeline,
\verb|direct_eval_cell| and \verb|interpolate_cell| stop at one iteration of 
interpolation and return $\tilde{\sigma}_{12}(z_\text{interp})$. Then,
\verb|fill_hypercube| mutates the $\tilde{\sigma}_{12}$ column of the original 
LHS by using the reverse transformation of~\ref{eq: scaling}:

\begin{equation}
x = \frac{x' - a}{b - a}
\end{equation}

to obtain the unit counterpart to $\tilde{\sigma}_{12}(z_\text{interp})$.
This transformed value replaces the original value found in the LHS.
Once \verb|fill_hypercube| finishes computing the spectra, the user
should save the LHS to a new file (marked ``final'' in
figure~\ref{fig: flow_chart}); this LHS is used in the emulator's training.

This shortcut saves a good deal of time but moves the value of
$\tilde{\sigma}_{12}$ negligibly
\textcolor{green}{cite some numbers for this claim!}.
The only downside of moving $\tilde{\sigma}_{12}$ could be that it
decreases the $s^*$ of the LHS. Since the value shifts only weakly, we do not
consider it to produce a relevant decrease in the accuracy of the emulator.

\textcolor{orange}{We should make error plots (comparing the two approaches 
with hyper cube as a control), but I strongly suspect the error will be 
negligible.}

In theory, the above procedure suffices to match any $\tilde{\sigma}_{12}$
value. However, for some cosmologies, $z_\text{exact} < 0$, a case not
currently supported by CAMB. Since all power spectra grow in amplitude with
time, we can quickly establish whether the cosmology is ``solvable'' by
verifying that

\begin{equation}
\label{eq: solvability_cond}
\tilde{\sigma}_{12}(z = 0) \geq \tilde{\sigma}_{12}(z_\text{exact})
.\end{equation}

When this condition is not fulfilled, we can modify the value of $h$ to
compensate. Recall from chapter~\ref{chap: A_s} that changing $h$ while
holding $\omega_b$, $\omega_c$, and $\omega_K$ fixed amounts to varying
$\omega_\text{DE}$, an evolution parameter. Since $h$ is an evolution
parameter in this context, we are free to modify it in order to increase the
range of solvable cosmologies. Although CAMB nominally accepts $h$ values as
low as 0.01, in practice we find that values below around
$h_\text{min} \approx 0.07$ sometimes lead to crashes.

We call this process of tweaking the $h$ and $z$ values `rescaling' to
succinctly distinguish it from the application of priors to scale a unit LHS.
In chapter~\ref{chap: emu_outline}, we noted that \texttt{ged} outputs
files containing rescale parameters. These files contain the $(h, z)$ pairs
at which we evaluated each cosmology in the input LHS. Since we train our
emulator over neither $h$ nor $z$, this information is inconsequential to the
construction and testing of the emulator. Instead, these files are useful for
independently verifying the accuracy of the pipeline.

Unfortunately, even when setting the dimensionless Hubble parameter to its
minimum safe value, some power spectra will still fail
condition~\ref{eq: solvability_cond}.
For the purposes of this treatment, we refer to such cosmologies as
``unsolvable.'' An emulator can still be trained even if only one of the input 
cosmologies is solvable. However, the occurrence of even one
unsolvable cosmology indicates that the actual range of parameters in the
training data is narrower than the nominal input priors. 

The problem of unsolvable cosmologies led us to create the
increasingly restrictive prior sets described in
section~\ref{sec: lhc_outline}.
Of the three provided pairs of priors files, only the COMET priors
(table~\ref{tab: COMET_priors}) are narrow enough to completely evade the 
issue of unsolvable cosmologies. For this reason, the COMET priors are the
defaults used by CL. We defer further discussion of unsolvable cells to
section~\ref{sec: prior_woes}.

%s Don't give up skeleton!


\section{Training the Emulator}
\label{sec: train_emu}

\textcolor{orange}{Don't forget to explain that the training X is actually the
unit LHS, not the rescaled LHS, because the unit LHS is already normalized
and therefore easier to handle in training.}

% train\_emu.py

Once \texttt{ged}'s \verb|fill_hypercube| completes, the user is ready to
begin training the emulator. We recommend that the user store all of the
relevant data files in a subdirectory of \verb|data_sets|, which is a
subdirectory of the \texttt{cassL} code directory. This organization
allows the user to load and repackage all of the essential data with the
\texttt{ui} function \verb|get_data_dict|, which returns the
``data dictionary'' represented in figure~\ref{fig: flow_chart}.
Once the user passes the data dictionary to the \texttt{ui} function
\verb|build_and_test_emulator|, the end of the emulator pipeline has been 
reached, and the remaining work is automatic.

First, \verb|build_and_test_emulator| cleans the input data by dropping
unsolved cosmologies from the training and testing sets. Next, to train a new 
emulator, \verb|build_and_test_emulator| instantiates the
\verb|Emulator_Trainer| class. We will refer to such objects 
as \textit{trainers}.

Each emulator is an instance of
the \texttt{Emulator} class with a \texttt{GPy} GPR object at its core. We 
will refer instances of the \texttt{Emulator} class as \textit{emulator 
objects}.

Besides
accessing this GPR object, emulator objects also convert $\bm{x}$ inputs and
$\bm{y}$ outputs in order to facilitate scientific access. Specifically, 
the GPR itself accepts $\bm{x}$ inputs and predicts $\bm{y}$ outputs 
in a normalized fashion which will not be easily understood unless first 
appropriately transformed.

We separate the \verb|Emulator_Trainer| class from the \verb|Emulator| class
to make saving and loading more efficient. When transferring an emulator to a
different computer, for example, it is not necessary to transfer all of the
data sets used.
% Although these data sets only turn out to be 2\% as large as the GPR object, 
% which is provided by GPy and which performs the predictions. Maybe the
% situation justifies my approach with the large-k data sets? I don't think
% so--I think the GPR object simply gets bigger to keep the proportion the
% same...
However, we still need the \verb|Emulator| class to hold more than just the 
GPR, for example the normalization parameters.



%s Now talk about normalization

\textcolor{orange}{Needs segue.}
The normalizations are necessary for the high performance of the emulators. 
This need goes beyond numerical instabilities. For example, if a parameter's 
prior range falls within the relatively reasonable [0.0001, 0.0006] interval, 
its emulation will still be improved by normalizing it to a [0, 1] interval. 
Nonetheless, the improvement will be more dramatic the further away a prior is 
from this [0, 1] interval. The prior in $A_s$, regardless of the choice
between the three default priors files, covers extremely small values. Consequently, 
without the appropriate normalization, the trained GPR will be nearly 
insensitive to the impact of $A_s$ and \textcolor{green}{almost behave like a 
massless-neutrino emulator}; the lack of neutrino dependence would dominate 
the error in the resultant emulator.
\textcolor{orange}{Great start! But can we add some material in
section~\ref{sec: gpr_intro}: why is GPR vulnerable in this way?}
 
\textcolor{blue}{This section will focus on the particular lines of}
\verb|GPy| \textcolor{blue}{that we used, as well
as the various data-cleaning and normalization statements that we used.
Normalization will be a really important topic. I want to explain some of
the theory behind why the emulator performs poorly with values outside of this
[0, 1] range.}

I don't know how to justify this kernel that I got from Alex via AndreaP...
\textcolor{orange}{That's no problem; Ariel says, just say ``we played around
and found that this works best.''}

\textcolor{blue}{Briefly explain what GPy is doing--don’t treat it like some black box.}


\section{Testing the Emulator}
\label{test_emu}

The majority of the quantification and visualization of the performance of the
emulator are left to the user. However, a couple of functions are provided for 
the sake of simple ``eyeball'' evaluation of the emulator's accuracy.
The \verb|test| function in the \verb|Emulator_Trainer| class (in the
\verb|train_emu| script) takes a set of $\matr{X}$ and $\matr{Y}$ test data 
and computes errors automatically. Specifically, it stores three sets of error 
metrics as class attributes for the trainer object: \verb|deltas| (the simple 
difference between the predicted spectra $\hat{\bm{y}}$ and the test spectra
$\bm{y}$, \verb|rel_errors| (the ratio of \verb|deltas| to the $\bm{y}$), and
\verb|sq_errors| (the squares of \verb|deltas|).

It follows from the definition of these error metrics that all of them have 
the same shape: each is a set of arrays equal in number to the $bm{y}$; and 
the size of each element array is determined by the number of scales at which 
the training power spectra were evaluated. \textcolor{orange}{In the current
version of the code, this number of scales must be equal in the training and
testing sets. But, if we add another layer of interpolation, we could easily
eliminate this limitation}.

The \verb|error_curves| function graphs the performance on the given test set 
using a collection of overplotted error curves, where each curve represents a 
single cosmology. By default, the $y$-axis is relative error and the
$x$-axis is the inverse scale, $k$. To break down the performance based on 
individual cosmological parameters, there is a \verb|param_index| function
parameter which will automatically color each curve according to its value in 
the chosen cosmological parameter. Since colors can sometimes be difficult to 
quickly compare, there is also a \verb|fixed_k| function parameter. When a 
fixed value for the inverse scale is provided, the plot becomes a scatter 
plot, with percent error as the $y$-axis and the chosen cosmological parameter 
as the $x$-axis.

%! Did I make sure to refer to priors correctly when transcribing ^ this 
%! function into train_emu.py? I think so… but it wouldn’t hurt to look one 
%! more time, especially since the X values are now automatically unit!

The \verb|error_statistics| function issues a rudimentary summary of the 
application of some NumPy aggregator\footnote{The user may also pass in
non-NumPy aggregators as long as they can be called in the exact same
way--the \verb|error_statistics| and \verb|error_hist| functions always
perform the call \verb|error_aggregator(errors, axis=1)|.}
\verb|error_aggregator|, which condenses each error array to a single point,
to one of the various error metrics described earlier. The 
most useful aggregators will be \verb|mean|, \verb|median|, and \verb|std|, 
but interesting information can also be gathered from, for example, the
\verb|min|, \verb|max|, and \verb|ptp| aggregators.
\verb|error_statistics| simply prints various statistics associated with the 
aggregation, such as the median of the aggregates.

The \verb|error_hist| function is essentially a visualization of the 
information printed out by the \verb|error_statistics| function: an aggregator
function is applied to the selected error array, then the aggregates are 
binned and plotted as a histogram. The user can specify the number of bins,
otherwise Sturges' rule is used.


\section{Accessing and Using the Emulator}

% user\_interface.py

\textcolor{orange}{Maybe we should rename this section, we're already doing
heavy access in the previous sections.}

\textcolor{blue}{The content of this section is still relatively uncertain. 
Since I am still tweaking some of the emulator's settings, I haven't spent too 
much time on a script dedicated purely to simplifying the interaction between 
the user and the emulator \textit{object} itself. Anyway, the hope is to 
provide some clear and simple 
descriptions of what functions the user should turn to in order to get started 
predicting power spectra using the results of this thesis.}

%s Now talk about the use of prior files

In section~\ref{sec: lhc_outline}, we introduced different priors 
included with CL, as well as their file names. These file names come into play 
when calling the function \verb|prior_file_to_dict|, which reads a prior file
into a Python dictionary. The reader is encouraged to add additional prior
files based on the format of the files provided in the \verb|priors|
subdirectory of the CL code.

%which is a helper function
% to \verb|get_data_dict|, which in turn is a helper function to the 
%\textcolor{orange}{unfinished} build_train_and_test_sets, which is a helper
% function to... ENOUGH!

  \chapter{Conclusion}
\label{chap: conclusion}

Massive neutrinos represent a challenge for evolution-mapping emulators
because their damping of the power spectrum depends on redshift.
This redshift dependence results in $\omega_\nu$ impacting both the
amplitude and the shape of the linear-theory cold-matter power spectrum.
This behavior precludes exclusive categorization as either a shape or
evolution parameter.

We find that two simple adjustments to evolution mapping can extend its
efficacy to the case of massive neutrinos. By recasting the evolution mapping
relation in terms of $\tilde{\sigma}_12$ rather than $\sigma_{12}$--that is,
in terms of the MEMNeC amplitude rather than the desired cosmology's
amplitude--we are able to circumvent the impact of $\omega_\nu$ on the
amplitude of the power spectrum, particularly on large scales. Furthermore,
we find that $A_s$ is not exclusively an evolution parameter but contains
information about the impact of massive neutrinos on the small-scale
power spectrum. By including it as a parameter over which we train the
emulator, we find that the small-scale accuracy significantly improves.

We introduce a new code, the Python package Cassandra-Linear, to construct
and test an emulator of the linear-theory cold-matter power spectrum based on
these extensions to the evolution mapping recipe. We offer a detailed survey
of the capabilities of this code so that interested readers can build and
experiment with their own emulators.

We compare the accuracies of two emulators trained using this code: one
including $A_s$ and $\omega_\nu$, and the other featuring no massive 
neutrinos. We find that the massless-neutrino emulator performs significantly
better when the number of samples is held fixed. Nevertheless, we show that
the $3\sigma$ confidence interval is below the 0.1\% error level. This means
the error of our massive-neutrino emulator is
comparable to that of CAMB, the Boltzmann solver we use to produce the
training spectra used for training.

We explore the impact of $s^*$, $N_k$, and $N_s$ on the performance of the
emulator and find...

In conclusion...

% Just use one big block of text, don't split into sections

\textcolor{red}{How do these results compare with results from 
papers on similar subjects?}

%s FUTURE WORK

To conclude this work, we identify several promising paths to advancing
both the Cassandra-Linear code as well as the scientific analyses in
chapter~\ref{chap: results}. The potential improvements are
numerous, but we here concentrate on the most important
suggestions.\footnote{For smaller suggestions, refer to the GitHub issues 
page: \url{https://github.com/3276908917/Master/issues}}.

%s Section 1: science improvements

It would behoove future inquiries to follow up on
sections~\ref{sec: error_from_lhc} and~\ref{sec: num_samples} by investigating 
whether we
can in some way compensate for the different impact of $N_s$ on the massive
and massless emulators so that their accuracies may be directly compared.
Such a comparison would help us to understand the error associated
specifically with the evolution mapping extensions introduced in
chapter~\ref{chap: A_s}. Even if a direct comparison method cannot be
established, it would be useful to know how much larger $N_s$ would have to
be for a massive emulator to reach equivalent levels of error.

% I can't touch this paragraph until the results are in
Based on the results of section~\ref{sec: error_from_lhc},
\textcolor{blue}{we do not consider LHC improvement to be a significant
priority}. But if it were... we would look through the theory. Can we build
a generator of perfect, or at least much  better, LHCs?

After finding that our emulator requires orders of magnitude less
time to produce a power spectrum than CAMB, we ceased to consider the precise
time cost. Instead, we compared different emulators exclusively on the
basis of accuracy. It would be helpful to compare the computation times of the
the various emulators. The large-$N_s$ samples, for example, lead to larger
emulators. The $N_s = 3000$ emulator is 219 MB, while the $N_s = 7000$
emulator is 1.5 GB. Do larger file sizes also mean slower emulators?

%s Section 2: code improvements. Segue offered by the uncertainty emulator.

The most important next step for the Cassandra-Linear code is the
extension of the emulator pipeline with a third data set, which we call the
\textit{validation} set. From the validation data we will build a validation
emulator, which will estimate the emulator error for any given cosmology.
We expect that combination of this information with the original emulator
predictions will achieve greater performance for the vast majority of
cosmologies. However, this will double the
computational cost both of building and querying the emulator.

We used a two-emulator approach analyzed in \ref{sec: 2emu_improvement},
which extended the range of applicability of our emulator. We can extend this
approach to account for discrete aspects of the cosmology. In particular, 
one could train two emulators of different mass hierarchies, inverted and
normal. The CAMB settings to change are in principle simple (see
section~\ref{sec: neutrino_settings}), but some work would need to be done to
see how the \texttt{mnu} parameter would need to be modified so that the
results are directly comparable with those cited here, which make use of the
degenerate mass hierarchy.

Our code would be more versatile if the user were allowed to specify a
distribution for each parameter in $\matr{X}$. For example, the emulator
performance appears to depend on $\tilde{sigma}_{12}$,
even if only for the trivial
reason that percent error depends on the amplitude of the true quantity.
Therefore, we were curious if the emulators would perform better by
sampling $\sqrt{\tilde{\sigma}}_{12}$ rather than $\tilde{\sigma}_{12}$.
In principle, the change should be fairly simple, but we encountered an
unknown error during the implementation and the feature is incomplete.

%s Section 3: use of different technologies

%s CLASS versus CAMB again

Although our priors should already suffice for most conventional parameter
inference studies, we believe that wider priors would be of benefit to those
seeking to understand exotic cosmologies. The chief obstacle to 
implementation of wider priors in CL is CAMB's requirement that $z \geq 0$,
which limits the range of cosmologies we can probe. 
Negative redshifts do not violate any conditions in the equations of
cosmological evolution, so it could prove fruitful to investigate why CAMB 
has this requirement; if the code could be easily amended, the results of
this work could be extended to broader priors. Alternatively,
CLASS \textcolor{green}{allows} negative redshifts, so it may be worthwhile 
to repeat the work of this thesis using power spectra from CLASS.

Lastly, we suggest the promulgation of quantitative results comparing the
performance of GPRs and neural networks. COMET is an example of a GPR-based
project that is switching to neural networks. It would be useful to consider
whether the emulators showcased here could be made more effective simply by
implementing a different machine learning setup. 

% Inquiries using different technologies

%s Code improvements

%%% Unprofessional to mention the following here:
\begin{comment}
The code should be expanded with documentation and unit tests. Also, the
user interface script is still in progress.

To simplify the user experience, this two-emulator solution lives ``under the
hood'' and by default \textcolor{orange}{will be} hidden behind an interface
which automatically queries the correct emulator given some user-input
cosmology.
\end{comment}
%%%

%%% Way too much detail for this thesis, but I'm glad that I at least thought
%%% about these things.
\begin{comment}
Our hope was that the narrow parameter ranges would furthermore help the 
demonstration emulator to achieve high accuracy--in principal, success here
means that we can simply ``scale up'' the approach of this work by
simultaneously expanding the priors as well as the total number of training
samples. Unfortunately, it is not clear if we can scale up the emulator past
the point at which unsolvable cells begin to appear. Since Latin hypercube
sampling is designed to evenly sample a space, unsolvable cells certainly
indicate that parts of the parameter space lack representation in the training
data. In these regions, our emulator will be forced to interpolate across
large gaps, or worse, extrapolate (if the unsolvable cells occur at the edges
of the parameter space \textcolor{orange}{This is something that I should have
shown... i.e. with plots}).
% Andrea recommends a plot coloring points by the "extremeness" in our sample
% space.

We predict that increasing the total number of cells will only marginally
reduce the issue of unsolvable cells \textcolor{orange}{This is something that 
I should have shown... i.e. with plots}). We can imagine the subspace of
solvable points as some hypervolume within a hypercube determined by our
priors, and the emulator's training coverage as an approximation of this 
hypervolume with small hypercubes whose size is determined by the separation 
between points in the sample. In the ideal case, the space of solvable points 
is the same as the Latin hypercube. When this is not so, we can at least
reduce the error associated with our approximation of the space of solvable
points by shrinking the hypercubes we use in our approximation (i.e. by
increasing the total number of points in our sample). \textcolor{orange}{
To give a sense of the marginal nature of this error reduction, we can
consider how small our hypercubes already are. For simplicity, let's examine
just one axis of the hypercube. With 5000 samples in the ``MEGA'' priors,
the length of the training coverage MOST STRONGLY DETERMINED BY ONE HYPERCUBE
IS: UNFINISHED THOUGHT}

It seems reasonable to think that unsolvable cells indicate extreme regions of 
the parameter space, rather than isolated holes. Therefore, it would be 
misleading to claim that the final ``MEGA'' emulator corresponds to, for 
example, any prior ranges in table 00A; in truth, the emulator would 
correspond to a potentially (this is a dangerous word and opens you up to hard 
questions) complicated shape inscribed within the six-dimensional rectangular 
hyperprism.
\end{comment}
%%%
  \chapter{Discussion and Conclusion}
\label{chap: disc_and_conc}

\section{Tightness of the Priors Used}
\label{sec: prior_woes}

With this section I would like to revisit the specific values for the priors,
that I only briefly mentioned back in section~\ref{sec: build_lhc}.

First of all, from a purely practical consideration, expanding the priors was
not feasible due to the high incidence of unsolvable cells.

But second, this may not be a significant limitation to the utility of the
emulators introduced here, because they are already quite wide compared to
current state-of-the-art parameter inferences. \textcolor{green}{CITATIONS}.

As of 19.06.23, ``COMET'' is the default for the emulator. It is the most 
restrictive of the three options and was implemented in order to totally 
eliminate the problem of unsolvable cells, allowing us to train our 
demonstration emulator over an LHS without any significant gaps.

Our hope was that the narrow parameter ranges would furthermore help the 
demonstration emulator to achieve high accuracy--in principal, success here
means that we can simply ``scale up'' the approach of this work by
simultaneously expanding the priors as well as the total number of training
samples. Unfortunately, it is not clear if we can scale up the emulator past
the point at which unsolvable cells begin to appear. Since Latin hypercube
sampling is designed to evenly sample a space, unsolvable cells certainly
indicate that parts of the parameter space lack representation in the training
data. In these regions, our emulator will be forced to interpolate across
large gaps, or worse, extrapolate (if the unsolvable cells occur at the edges
of the parameter space \textcolor{orange}{This is something that I should have
shown... i.e. with plots}).
% Andrea recommends a plot coloring points by the "extremeness" in our sample
% space.

We predict that increasing the total number of cells will only marginally
reduce the issue of unsolvable cells \textcolor{orange}{This is something that 
I should have shown... i.e. with plots}). We can imagine the subspace of
solvable points as some hypervolume within a hypercube determined by our
priors, and the emulator's training coverage as an approximation of this 
hypervolume with small hypercubes whose size is determined by the separation 
between points in the sample. In the ideal case, the space of solvable points 
is the same as the Latin hypercube. When this is not so, we can at least
reduce the error associated with our approximation of the space of solvable
points by shrinking the hypercubes we use in our approximation (i.e. by
increasing the total number of points in our sample). \textcolor{orange}{
To give a sense of the marginal nature of this error reduction, we can
consider how small our hypercubes already are. For simplicity, let's examine
just one axis of the hypercube. With 5000 samples in the ``MEGA'' priors,
the length of the training coverage MOST STRONGLY DETERMINED BY ONE HYPERCUBE
IS: UNFINISHED THOUGHT}

It seems reasonable to think that unsolvable cells indicate extreme regions of 
the parameter space, rather than isolated holes. Therefore, it would be 
misleading to claim that the final ``MEGA'' emulator corresponds to, for 
example, any prior ranges in table 00A; in truth, the emulator would 
correspond to a potentially (this is a dangerous word and opens you up to hard 
questions) complicated shape inscribed within the six-dimensional rectangular 
hyperprism.

\section{Minimum Separation of the Training LHC}
\label{sec: error_from_lhc}

What is the impact of the minimum separation? Surely the minimum separation
should be a proxy for the evenness of the coverage of the space of
cosmologies. Therefore, we expect the error variance to increase much more
dramatically than, say, the average bias.

How would we
be able to quantify the error due to this? We could try to compare the
emulator performance trained on hyper cubes of various minimum distances.

\section{Resolution of the k Axis}

This might go better in the CassL section, but I think I ought to motivate the decision to use length-300 arrays.

\section{Number of Training Samples}
\label{sec: num_samples}

\textcolor{blue}{Justify choice of 5000 samples for each: maybe we can make a
trend plot showing diminishing returns in test error?}

% This might go better in the CassL section, but I think I ought to motivate the decision to use 5000 training arrays.

\textcolor{orange}{I'll have to concede that the results of this section are not entirely comprehensive; we didn't train any emulators over the uncertainties of analogous validation hypercubes. All comparisons here use the simpler pipeline of just two data sets, training and testing.}


\section{Linear Sampling in Different Parameters}

We also tried sampling in $\sigma_{12}^2$ as well as $\sqrt{\sigma_{12}}$.
Unfortunately, we were unable to conclude anything about the effectiveness of
these strategies--there appears to have been some mistake in our code, such
that the errors are much larger than can be explained on account of poor
sampling.

See figures~\ref{fig: sigsquare_sample} and~\ref{fig: sigroot_sample} for
illustrations of the problem. In a future work, it would be helpful to
investigate these problems further. We may find that a different sampling
strategy will more efficiently reduce the deltas that we see in our emulator.

\section{Summary of the Paper}

What was the main objective of this thesis? What were the key results of this work? Why are they important? How do these results compare with results from papers on similar subjects?

\section{Future Work}
\label{sec: future_work}

Here I will talk about what kinds of questions we estimate will be most fruitful for further inquiries about this topic and this code.

Is there a theoretically perfect LHC generator?

In order to expand the priors, it would be helpful to investigate why CAMB does not allow negative redshifts, in case this can be adapted. Alternatively, CLASS \textcolor{green}{allows} negative redshifts, so it may be worthwhile to repeat the work of this thesis using power spectra from CLASS. This would involve familiarization with a new platform, however, and so this escapes the scope of this thesis. Remember that negative redshifts would be a helpful feature for this work because it would allow us to investigate much broader priors.

Should we use a neural network instead of a Gaussian process for our emulator? AndreaP and Alex Eggemeier are already on the job: they are converting COMET to a neural network approach. We recommend that the reader follow future COMET papers for investigations into this question.

GP's allow natural propagation of
uncertainty in predictions to the final posterior distribution; neural
networks lack this feature. At the same time, NNs provide larger speedups \textcolor{green}{CITATIONS}.

  \include{anhang}


  \backmatter
  \include{bibliographie}
  \markboth{}{}


  \addcontentsline{toc}{chapter}{\protect Danksagung}


\chapter*{Danksagung}

We thank Daniel Farrow for his work during the CAKE 2021 workshop, which
served as the basis for the vital lhc.py script in the Cassandra-Linear
package.



  \include{lebenslauf}


\end{document}