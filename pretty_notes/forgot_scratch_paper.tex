\documentclass[11pt]{article}
\usepackage{reports}

\newcommand*{\instr}{Ariel S\'{a}nchez}
\newcommand*{\term}{30.11.2022}
\newcommand*{\coursenum}{Master's Thesis}
\newcommand*{\hwnum}{Treatment of Neutrinos within Evolution Mapping Schemes}

\usepackage{pdfpages}
\usepackage{bm}
\usepackage{listings}

\usepackage{titling}
\usepackage[normalem]{ulem}

\graphicspath{{./res/}}

\begin{document}

\fontsize{12}{15}

Welcome to the pretty notes. Here we can record observations on papers, but
only if I didn't have any paper around. This is also a good place for to-do's,
and terminology that comes up in weekly group meetings.

\section{To-do}

\$ Explain the master thesis situation to Ariel.

\$ Respond to administrative emails and fill out forms as necessary until you
have a computer account, internet access, a keycard, and a nametag. (Right now
we're waiting on Frau Kestler to provide a pre-signed contract).

\$ Clone the master thesis repo to this computer

* Begin crunching your notes
into a master thesis write-up. That is to say, you can begin to write up the
introduction and background components to your final report.

* Reproduce the plots that Ariel showed you. To recapitulate, we want ratio
plots $P_\nu / P_0$ (each with a few different lines) where each $P_\nu$ line
represents a different massive-neutrino cosmology and the corresponding $P_0$
line represents the same cosmology except for the absence of massive neutrinos.

To keep the $\omega_\text{matter}$ values consistent within each pair of lines,
you will need to define some $\omega_c$ for the massless neutrino models which
picks up the mass lost by depriving the neutrinos of mass.

The $\sigma_{12}$ values also need to be consistent within each pair. How do
you decide the value? (Maybe we could let the massive neutrino cosmology
determine the baseline amplitude since it's counterpart massless cosmology is
engineered anyway). Anyway, because the $\sigma_{12}$ values need to be the
same, the $z$ value must change--specifically, you should vary the redshift
stamp of the power spectrum until both the massive-neutrino and
massless-neutrino models share a value of $\sigma_{12}$.

One set of ratios was like a flat line but with a perhaps 0.1\% variation about
the central line. The other set of ratios was much more interesting, it started
flat but at increasing values $k$ (these were all logarithmic plots on the $x$
i.e. $k$ axis, but linear on the $y$ i.e. ratio axis) it began to drop off,
with a minor oscillation appearing as well. \textcolor{red}{What was the
difference between the two plots?}

I'm less certain about this, but I think that each pair of lines in each plot
was characterized by some unique neutrino mass.

% If I am the last person to leave, I ought to lock the doors.

% Don't forget to close the door when you re-enter. It's what they've always
	% done.

% We should be counting to TJoInx4r but for some reason we get stuck at
% 20EoG419
% alpha beta psychedelic funking

% We should clean up our area and start heading out at the following times:
	% Tuesday: 1640
	% Every other day: 1740

\section{Workplace Productivity}

This paragraph was written at 11:35 am, 18.10.22. I entered the institute at
around 11:10 am.

To better prepare for Garching visits:

* \sout{Scratch paper}

* \sout{Dedicated mouse (we need to search around for this)}

\section{Lessons in Appraisal}

    If something funny is happening only with models 1 and 2, then you should
    strongly suspect that your conversion away from $h$ units has a hiccup
    somewhere.

    \subsection{Ratio of massive and massless power spectra}

        If the curves for the different models are not appropriately aligned,
        but instead appear with a sort of k-axis offset, check your $h$
        conversion for the k-axis. Intuition: since you're taking ratios, you
        don't need to worry about h mistakes in your y-axis (i.e. the cubic
        $h$ factors in the power spectra already cancel themselves out).
        Therefore, the error lies in the k-axes for the curves of different
        models being scaled by different $h$ values.

        If you're comparing curves at one red-shift (that is nonzero!) and the
        curves seem to blur out a little (i.e. the different models slightly
        but fairly universally disagree), then double-check that you are
        comparing like redshifts to like. If $z \neq 0$, then $z$ will have
        to change from one model to the next if you want to control for
        amplitude.

        If your curves at any point exceed unity, you are getting something for
        nothing. In my particular case, that  meant that I was subtracting CDM
        physical density without adding it back in in the form of neutrinos.
        Conceivably, this principle could also be used if the power spectra at
        large scales also \textit{under}shoots unity, for example if I add
        neutrinos without subtracting CDM physical density.

        If the curve shape at the lowest scales disagrees dramatically, there
        is a good chance that you are computing the non-linear power spectrum,
        when we're truly interested in the linear power spectrum.

    \subsection{Suspicions that Require More Testing}

        When I switched from the degenerate to either the normal or inverted
        neutrino mass hierarchy, the ratios of massless and massive power
        spectra exhibited behavior reminiscent of the k-axis $h$ misconversion.
        Not identical, though: the curves were also just barely-visibally
        shifted along the y axis. Is this perhaps an unambiguous sign of
        hierarchy disagreement? If so, what could possibly be the intuition
        behind the behavior of the plot?


\section{Meetings: Principles}

``When in doubt, try re-binning.''

Ariel says you should put math in blue so as to project confidence. (In
general, it seems like coloring a presentation is a polarizing topic!)

When making a presentation, make sure not to take our conventions for granted.
For example, we use physical units and we tend not to use $h$ units, but
``nobody else'' does these things, so take your time to introduce these
conventions when you are presenting to people outside of our OPINAS group.

``Each bullet point should precede at most two lines of text.''

\section{Glossary}

$A_s$: ``scalar mode amplitude''

Bootstrapping: ``any test or metric that uses random sampling with replacement
(e.g. mimicking the sampling process). Bootstrapping assigns measures of
accuracy (bias, variance, etc.) to sample estimates. This technique allows
estimation of the sampling distribution of almost any statistic using random
sampling methods'' (Wikipedia).

Cosmic variance: typically refers to the affect of cosmic large-scale structure
on measurments such that a region of sky as viewed from Earth ``may differ from
a measurement of a different region of the sky also viewed from Earth by an
amount that may be much greater than the sample variance (the difference
between different finite samples of the same parent population.)'' Less
frequently, it can describe the issues associated with having access only to
one realization of the Universe (Wikipedia). 

CPL parametrization: $w_\text{DE}(a) = w_0 + w_a (1 - a)$.

Degenerate hierarchy: a neutrino parametrization in which all
neutrino masses are much larger than $\sqrt{\Delta m_{ij}^2}$. I'm no expert,
but I'm pretty sure that this just means that all (usually three) masses are
larger than
the differences in mass between any two neutrinos.

The dimensionless power spectrum: $\Delta_L^2(k) = \frac{k^3 P_L(k)}{2 \pi^2}$.

$D(z)$: see linear growth factor.

$f(z)$: see logarithmic growth rate.

Growth factor $D(z)$: see linear growth factor.

Growth rate $f(z)$: see logarithmic growth rate.

HETDEX: "the first major experiment to search for dark energy in the
early Universe. Using the giant Hobby-Eberly Telescope at McDonald Observatory
and world-class instruments the team will map the three-dimensional positions
of millions galaxies to help explain expansion over time."

HOD: Halo Occupation Distribution. The HOD is a parameter of the halo model of
galaxy clustering, which views the large scale structure of the universe as
clumps of dark matter. In this model, HOD specifically refers to the
distribution of galactic matter within each of these DM clumps (Wikipedia).

Jackknife resampling: a cross-validation technique. ``It is especially useful
for bias and variance estimation. Given a sample of size $n$, a jackknife
estimator can be built by aggregating the parameter estimates from each
subsample of size $(n - 1)$ obtained by omitting one observation. It is a
linear approximation of the bootstrap'' (Wikipedia).

\textcolor{red}{Kaiser boost? It appears at large scales?}

Kaiser factor:

The Kaiser factor is evaluated as $P_{g, \ell=0} (k) / P_g$, where
$P_{g, \ell=0}$ is the isotropic (i.e. monopole) component of the Kaiser
formula ("Galaxy Clustering in Redshift Space," Saito, 2016).

Kaiser formula:

The Kaiser formula gives the RSD correction at linear order:

$\delta_m^{s, L} (\bm{k}) = (1 + f \mu^2) \delta_m^L (\bm{k})$

``The redshift-space power spectrum at linear order is given by:''

$P_m^{s, L} (\bm{k}) = P_m^{s, L} (k, \mu) = (1 + f \mu^2)^2 P_m^L (k)$

``In the case of galaxy number density with $\delta_g = b \delta_m$, similarly
one obtains:''

$
P_g^{s, L} (\bm{k})
=
P_g^{s, L} (k, \mu)
=
b^2 (1 + \beta \mu^2)^2 P_m^L (k)
$

(ibid.)

$k_p$: see pivot scale

    Linear growth factor $D(z)$: modifies the initial density field as a
function
of redshift, i.e. time elapsed since the initial density field represented
the actual density field. From Kiakotou et al 2008: ``the growth factor
affects the matter power spectrum by changing the amplitude.''

Logarithmic growth rate: $f(z) = \frac{d \ln D(z)}{d \ln a}$

Neutrino mixing: the neutrino mass and flavor eigenstates mix and give rise
to neutrino oscillations. Conceptually, this mixing is the same as  one would
expect behind any observable associated with the wave function of some quantum
system.

$n(M)$: the halo mass function

Pivot scale $k_p$: ``the power spectrum of primordial density perturbations is
often modeled as a power law, $P(k) = A \left( \frac{k}{k_0} \right)^{n - 1}$.
This is the form predicted by slow roll inflation and it fits current CMB and
LSS data well. Here, $k_0$ is the pivot scale. It is the scale where the
amplitude A is measured. It is possible to include so-called `running,' in
which the spectral index, $n$, itself has a $k$-dependence
$\alpha = \frac{dn}{d \ln k}$, so that $k_0$ serves as the scale on which the
spectral index is measured'' (`bapowell,' Stack Exchange June 28 2020).

Resampling: the creation of new samples based on one observed sample.

Rocket effect:

``LG motion can produce a spurious effect on clustering measurements of the
Universe’s Large Scale Structure (LSS) when traced by the galaxy distribution.
It is related to the Rocket Effect, where the local group motion can induce a
spurious apparent overdensity in the direction of motion, which then may appear
to be the cause of the motion in the first place. In principle, the Kaiser
Rocket effect should not be neglected and can be corrected if the selection
function is sufficiently well-known. A recent investigation on the wide-angle
correlations to the galaxy power spectrum in redshift space suggested that the
Kaiser Rocket effect could dominate the local signal of the 2-point correlation
function of galaxies at very large scales'' (Bahr-Kalus et al, 2021).

s: separation

Sterile neutrino: hypothetical particle interacting only gravitationally, and
in no way through the weak force. If I recall correctly, this is a dark matter
candidate.

``The Transfer Function $T(k)$: relates the processed power-spectrum $P(k)$ to
its primordial form $P_0(k)$ via $P(k) = P_0(k) \times T^2(k)$.'' That is to
say, $T(k) = \sqrt{\frac{P(k)}{P_0(k)}}$. For more details, see section 2.4
from Peter Coles' delightful review:
\url{https://ned.ipac.caltech.edu/level5/March01/Coles/Coles2.html}

Voronoi diagram, tesselation, decomposition, partition: the partitioning of a
plane of points based on a subset of the points which act as seeds. Then the
shape / extent of each Voronoi cell is such that it contains all the points
closer to its seed than any other seed (Wikipedia).

Watershed algorithm: a transformation defined on a grayscale image. The
brightness of the pixels are viewed in a topological sense, and one attempts
to divide the image into basin regions by tracing out the ridges formed by
the brightest pixels (Wikipedia). This technique didn't work in void
computation because it returned that the entire Universe is a void (Carlos).

Wetterich early dark energy model:
$w_\text{DE} (a) = \frac{w_0}{[1 - b \ln(a) ]^2}$

where

$
b = \frac{3w_0}{
\ln \left[ \frac{1 - \Omega_\text{DE, e}}{\Omega_\text{DE, e}} \right]
+
\ln \left[ \frac{1 - \Omega_m}{\Omega_m} \right]}
$

\section{``Redshift-Space Distortions, Pairwise Velocities and
Nonlinearities''}

Peculiar velocities are sourced by gravitational potentials, and gravitational
potentials are mostly sourced by dark matter. Therefore, RSD observations are
measurements of dark matter content in the Universe.

\section{Ariel S\'{a}nchez's FECS, Chapter 1}

The covariant derivative $\nabla_\mu$ is defined as:

\begin{equation}
	\nabla_\mu V^\nu = \partial_\mu V^\nu + \Gamma^\nu_{\mu \rho} V^\rho
\end{equation}

and

\begin{equation}
	\nabla_\mu Y_\nu = \partial_\mu Y_\nu - \Gamma^\rho_{\mu \nu} Y_\rho
\end{equation}

``The metric contains all the geometric information of the manifold'' p. 13.

At the bottom of page 13 it is stated that metric compatibility implies
$\nabla_\rho g_{\mu \nu} = 0$ and $\nabla_\rho  g^{\mu \nu} = 0$.
\textcolor{orange}{Isn't this trivial? I mean, the metric must be a constant
tensor object since we are using it to continually refer to the same manifold.
How could the derivatives of a constant tensor object ever be anything but
zero?} \textcolor{red}{No! (So you must learn this concept. Maybe when we have
internet again we can look up metric derivatives.)} See this quote from the
following page: ``It is possible to show that on every point in a given
manifold it is possible to find a coordinate system in which the first
derivatives of the metric tensor vanish at that point.''

By convention, $\mathbb{R}^2$ and $\mathbb{R}^3$ refer to spaces without
curvature--that is to say, they are implicitly subject to Euclidean geometry.

It is difficult to gauge the intrinsic curvature of a surface from the metric
tensor because non-constant terms could be artefacts of the coordinate system
used. For intrinsic curvature, we examine the Riemann curvature tensor instead.
All of the Riemann tensor terms vanish every iff the space is flat.

Under the Lorentzian sign convention, the definition of proper time is

\begin{equation}
	d\tau^2 = -ds^2
\end{equation}

\textcolor{red}{Although I don't quite understand this definition. Take, for
example, the twin paradox. The twin who never accelerated measured much more
time. The twin who accelerated twice measured much less time. So the
accelerating twin's trajectory was spatially much longer--shouldn't that extra
length feed back into the proper time under this definition? Basically, I'm
saying that my intuition tells me that more time passes with trajectories
involving less spatial position change, as long as we hold the amount of time
passed as fixed.} The resolution cannot have anything to do with the separation
infinitesimal being timelike or spacelike, because we square it.

Simple mnemonic: for any timelike separation vector, the passage of time
contributes the dominant component of the vector length.

The energy of a particle is given by the timelike component of its
four-momentum, $p^0$. In the particle rest-frame, $p^0 = m$.

\end{document}
