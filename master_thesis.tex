\documentclass[11pt]{article}
\usepackage{reports}

\newcommand*{\instr}{Andreas Burkert}
\newcommand*{\term}{15.08.2022}
\newcommand*{\coursenum}{Gravitational Dynamics}
\newcommand*{\hwnum}{Hausarbeit}

\usepackage{pdfpages}
\usepackage{bm}
\usepackage{listings}

\usepackage{titling}

\graphicspath{{./res/}}

% One plot per week requirement
% You have two weeks after the semester ends,
% in order to finalize the Hausarbeit.

\begin{document}

\fontsize{12}{15}

\begin{center}
Lukas Finkbeiner: Master's Thesis
\end{center}

This is a master's thesis. I hope to defend it to your satisfaction.

\textit{What is a Boltzmann solver? What is the MCMC?}

In order to constrain cosmological parameters based on the application of this method to \textcolor{red}{LSS data sets} [vague], we must call a Boltzmann solver for each point in our parameter space. This multitude of calls represents an enormous computational, and therefore time, cost.

Emulators have been recently proposed as a solution to this bottleneck. \textit{What is an emulator? How do we train it?} (Arico et al. 2021, Mancini et al. 2021).

These emulators sample parameters organized into two categories: purely evolution parameters, and paramateres that affect both the evolution and the shape of the power spectrum. \textcolor{orange}{Is there no pure shape category? Why does it help to have these categories in the first place?} Conventional emulator calibration entails the historical units of Mpc / $h$, but if we use instead units of Mpc, then we can 

From Kiakotou 2008: ``Neutrinos with mosses on the eV scale or below will be a hot component of the dark matter and will free-stream out of overdensities and thus wipe out small-scale structures.''

\end{document}